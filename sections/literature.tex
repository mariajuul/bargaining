%Mincer 1978 - static
%Lundberg Pollak 2001 - dynamic
% Gemici 2011 - dynamic
% Mazzacco - dynamic
% Costa Khan

This section gives an overview of the literature on household location decisions. First off, however, unitary and non-unitary models are briefly explained. Next, I go over static models from the literature and end with dynamic models.

\section{Unitary vs non-unitary models}\label{sec:uni_nonuni}
Household decision studies have originally used unitary models (\cite{Samuelson1956,Becker1962,Becker1981}) where the analysis treats the household as a single individual with one utility function and it cannot identify individual utility functions. It thus cannot predict how an individual will react in response to changes in a certain policy, only how the household as a whole will. The model has been criticized for not capturing how the actual decision process in a multiple-member household takes place and some of the first attempts to deal with this were \cite{ManserBrown1980} and \cite{McElroyetal1981} who introduced the first non-unitary models (for the basic theory of these models, cf. \cite{Chiappori1988,Chiappori1992,BrowningChiappori1998}). Unlike unitary models, non-unitary models deal directly with separate utility functions for each spouse and these models fall into two broad categories: cooperative non-unitary models and non-cooperative non-unitary models. The seminal papers by \cite{ManserBrown1980,McElroyetal1981} both belong to the cooperative framework where the household behaves as if it is maximizing a weighted sum of the spouses' utility functions, where the weights can depend on both prices, wages and distribution factors - variables that do not enter preferences or the budget constraint but affect the bargaining power, e.g. the income ratio of the two spouses.  The bargaining process over outcomes between the spouses lead to Pareto efficient outcomes. This is in contrast to non-cooperative models, introduced by \cite{LundbergPollak1993}, where Pareto inefficient outcomes may occur, i.e. where the household could have chosen differently and made at least one spouse better off without making the other spouse worse off. In this framework the household decision process is modelled as a game where each spouse maximizes his or her own utility while taking as given the decision of the partner. 

There is no clear answer to whether cooperative or non-cooperative models are better at explaining household behaviour. \cite{Udry1996} rejects Pareto efficiency while \cite{Bobonis2009} does not. The empirical tests for the unitary against the non-unitary models are carried out in studies analysing mainly consumption by testing whether income pooling holds (\cite{Lundbergetal1997,AttanasioLechene2002}, among others), i.e. household decisions do not depend on whom of the member receives the income, and generally reject the unitary models. 
%\cite{Browningetal2006}, however, underline that a more proper definition would be that unitary models are those where the demand satisfies the Slutsky conditions (adding up, symmetry, homogeneity and negativity of the Slutsky matrix) - and may or may not satisfy income pooling. 

Generally, it has been recognized that it is important to account for the fact that many decisions within the household result from multiple agents reaching an agreement. Ignoring this and instead regarding the household as if it were a single individual will likely lead to models with biased estimates since it is trying to rationalize a decision process characterized by two individuals' utilities and not just one (see \cite{Picardetal2013} for a discussion of non-unitary models in urban economics). 

\section{Static models}
The literature on household location decisions that takes into account this two-person structure of the households dates back to \cite{Mincer1978}. He introduced the concept of a tied family member - the one who compromises on his or her individually optimal moving decision and rather moves or stays because the partner's gain from doing so outweighs his or her disadvantage such that the family as a whole gains from that decision. He refers to this as negative personal externalities which may or may not be internalized by the household unit. He also points out that with the, at the time, growing labor force participation of women, both the wife and the husband might become tied because they both might gain from living (and in this case also working) somewhere else. The process underlying the location decisions of couples therefore can differ from that of singles.

\cite{CostaKhan2000} takes up Mincer's tied mover and stayer concept and use a reduced-form model to analyse whether power couples - couples with two college-educated spouses - are increasingly likely to locate in big metropolitan areas, as observed in US data, because of the colocation problem they face. Namely, that due to both partners' specialized skills and their active labor market participation, they find the thicker labor market in the cities relatively more attractive than part-power couples (couples with only one college-educated spouse), singles and low-power couples (couples with no college-educated spouse) who only look for one or zero job matches for specialized skills. This explanation is compared to the explanations that power couples are over-represented in these urban areas because they can share the cost of the high rents, returns to education in relative large cities has risen compared to small cities or because urban amenities are normal goods. They conclude that the colocation problem indeed is the most likely explanation. \cite{FreedmanKern1997} reaches a similar conclusion, namely that when both the wife and the husband have a professional career, they are more likely to reside and work in big cities and that wives' differing earnings potentials across the country (here: US) affect the chosen home location of the couple and thereby also the husband's work location. Other studies focus more on the commute times of each person in the household, e.g. \cite{SermonsKoppelman2001}, and generally find that household decisions are more sensitive to increases in the wife's commute time. Since the effects are more pronounced for families with children they conjecture that this is due to the wife having more household responsibilities. 

Modelling the actual bargaining process in location choice models is nevertheless still in its infancy. \cite{Chiapporietal2014} is an exception and is one of the first that do an attempt to account for the effect that individual-specific characteristics of members in the households have on decisions - both through affecting the bargaining power and via the individual's preferences for certain alternatives. They estimate a static multinomial Logit model for residential areas in France conditional on each spouse's workplace and find that taking bargaining power into account is important in order to get unbiased estimates of the value of time. Additionally, they find evidence that residential location choices are Pareto optimal which speaks in favor of using collective models as concluded in section~\ref{sec:uni_nonuni}.

\section{Dynamic models}
In addition to the above considerations of the structural differences in the decision-making process between single- and multiple-person households, a discussion of dynamic vs static models within this regime has taken place in recent years. Such dynamic aspects are important when the researcher wants to evaluate policies with intertemporal dimensions. 

Intertemporal unitary models can be, and have traditionally been, used to study how households allocate scarce resources to different purposes (e.g. income to different goods, time to labor, household production and leisure) and the intertemporal allocations of these (\cite{Scholzetal2006,KruegerPerri2006}). Just as the static unitary model, the dynamic unitary model is not suitable for studying allocation of goods across spouses and moreover assumes bargaining power is constant across households and time.

When it comes to intertemporal collective models there are generally two types: those with limited (LIC) and those with full commitment (FIC), cf. \cite{ChiapporiMazzocco2015} for a thorough review of these models. The latter are those where the households can commit to future allocations, i.e. households formulate a plan of optimal decisions at the beginning of marriage and stick to this plan no matter the shocks they might experience during the course of life. The LIC models are more complex and have higher data requirements but relax this assumption by requiring households to make efficient decisions subject to the participation constraint of both spouses, i.e. that both spouses must be at least as good off in the marriage as if they take their best outside option, typically the value of a divorcee. It also allows households to renegotiate the plan of allocations in cases where the particiaption constraints are no longer fulfilled due to changes in the outside options. Formally, the FIC is the LIC without the participation constraints. By extending the static framework of the collective model in \cite{Chiappori1988,Chiappori1992}, \cite{Mazzocco2007} proviodes empirical evidence that the LIC is favoured over FIC in a study of consumption and savings. Similar conclusions are reached in \cite{Aura2005} and \cite{LiseYamada2014}.

 Even though the dynamic collective framework has become more popular, it has not been used very much in the location choice literature. \cite{LundbergPollak2003} point out from a theoretical viewpoint that choice of residence for couples is indeed a collective decision where bargaining power matters. Also, they argue that these decisions need not be Pareto efficient since one spouse might veto a move due to the expectation that this will deteriorate his or her bargaining position. This could be due to lower earnings potential in that area which lowers the outside option. Households therefore might find themselves in a situation where they could Pareto optimize but do not do so because of lack of commitment of staying in marriage and sharing the household income according to the current sharing rule. \cite{Gemici2011} is so far the only paper that estimates a dynamic collective model of couples' residential location decisions. She employs a Nash bargaining framework where couples choose consumption, location, employment and divorce each period taking into account their outside options and concludes women are more likely to be the tied spouse. However, by assuming transferable utility the household is able to compensate the tied spouse such that all decisions are efficient. 
% mention in model: However, the above discussion and cited papers concern continuous and static optimization problems. So far there is not much empirical evidence on dynamic discrete choice models