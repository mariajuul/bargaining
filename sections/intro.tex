% Why
% What
% How
Since women started to really join the labor force, the dual-earner household has been representative for many couple households in the Western part of the world. When such households decide where to live they face a co-location problem where both spouses' work locations are considered. The trade-off considered in this paper is the one that occurs when the home location is made togehter with job locations of the spouses. The idea is that an individual spouse's current job (location) affects his future job opportunities, e.g. because working in a low quality job location means the human capital depreciates and thus makes it harder for him to get a high-salary job in the future. One spouse might therefore dislike living in a certain area if that implies only poor jobs matches can be reached within a fair commute time, whereas the other spouse with another skill set might prefer that area. In the literature on household location decisions however the agent has typically been a household representative that ignores the dual-earner framework and rather assumes households can be characterized by a single utiltiy function in a unitary model or in a model of individual decision makers. Given Arrow's impossibility theorem which states that it is indeed impossible to aggregate preferences in a reasonable way, modelling household decisions more accurately calls for explicitly modeling preferences of each household member. 

More specifically, the research questions of interest are how politicians can avoid the depopulation of more rural areas. This is of particular interest to the Danish government who recently implemented the policy ``Bedre Balance'' (in English: Better Balance) where it moved about 3,900 public sector jobs outside the capital area in Copenhagen where they have been located traditionally\footnote{See \autoref{apfig:betterbalance} for an overview of the location of the moved jobs.}. These jobs are mainly academic jobs and a significant number of the jobs were moved to rural areas. The question is whether couple households actually do respond to a policy like this. Are they willing to move to the rural locations due to more jobs being available or will it only affect people who do not yet live in a couple? Many couple households do need to be able to get two jobs after all. Will people rather be more willing to commute to these areas because they can now earn a better wage there so the commute is worth it but still reside in their current location? Also, how does the willingness to move depend on the type of household and could the development of rural areas be achieved in another way for instance by building better and faster public transport between these areas and the urban areas of Denmark?

To answer these questions I break with the unitary and indiviual modelleing tradition by using a collective model of cohabiting and married couples' decisions on home and job locations. I incorporate the collectiveness in a dynamic structural model by giving each partner a bargaining power that determines how much weight the household attaches to the two individual current and future value functions. The two-person household chooses joint home location and separate work locations each period taking into account commute time and wages for each spouse, amenities of the home region and the effects that the current decisions have on future job prospects and bargaining power. 

Collective models can be used to study the welfare effect on each household member from changes that otherwise only seem to impact one household member. The welfare gain from improving public transport may be higher in collective models because the member that does not directly benefit still does so because the household would not have to move in order for the affected spouse to realize the better job further away from the current location. Thereby the other spouse does not suffer in terms of giving up a currently well-paid job or from spending time on applying for new jobs. To the best of my knowledge, \cite{Gemici2011} is the so far only paper that estimates a structural dynamic model of two-person household's location decisions by taking the bargaining into account. However, she does not distinguish between home and work locations and therefore cannot let spouses commute in order to obtain get a good job. 

For estimation I use high-quality Danish administrative data where I am able to track the entire population of households and its members from birth to death from 1991-2015. Introducing bargaining into a dynamic model is non-trivial and this together with the large state space that the rich dataset allows for make full-solution methods infeasible for estimation. Instead, I aim at using the CCP methods developed by \cite{ArcidiaconoMiller2011}. The structural estimation has yet to be implemented.

The rest of the paper is organized in the following way: \autoref{sec:lit} gives an overview of the existing literature on household decisions that take into account the two-person structure of the family. Section~\ref{sec:data} explains the content of the data used in this paper and \autoref{sec:descriptives} shows descriptive statistics used to assess the determinants of location choices and how couple and single households may differ in this respect. Next, \autoref{sec:model} introduces the structural model suggested to model the location choices. However, this is only a tentative suggestion and modifications to this model may take place if it turns out to not explain the data well. Section~\ref{sec:results} has yet to be done and will present the results from the structural estimation of the model. Section~\ref{sec:conclusion} concludes.

% \section{Suggested research questions}
% \textit{Currently not sure exactly what the research question should be but here are some suggestions:}

% \subsubsection{It's wrong to use a unitary model - what's the effect of using collective model?}
% When modeling the location decision as an individual optimization problem the model may have problems matching data on dual-earner households since it does not take into account that what we see in the data is not necessarily the optimal decision from the individual's viewpoint. Another likely consequence of restricting attention to individual optimization problems is that it may distort the estimates of certain parameters. Especially moving costs since the model would infer that it must be very costly to move since the individual did not do so even though the prospects would look much better in another area when it is actually just because the person is tied by his or her partner's job location. Previous papers in the literature have tried to control for the presence of a spouse by including a dummy for this or variables measuring household size in the moving cost specification. However, the parameter often turns out negative indicating that larger families (e.g. dual-earner households) have lower moving costs than singles do. This is presumably in order for the model to explain that couples make seemingly non-optimal moves when looking at the optimization problem from the individual's point of view. This calls for modeling the family optimization decision explicitly. 

% \subsubsection{Do couples tend to locate more in cities compared to singles due to the dual-earner constraint? What is the effect of infrastructure investments?} 
% Do households who decide on optimal locations by trading off commuting costs, wage potentials and amenities of different locations choose differently if commuting was faster between rural and urban areas? The challenge for these households is that one spouse may be more constrained than the other in terms of finding a good job match in certain locations and this tends to constrain the other spouse as well. The non-tied spouse therefore still becomes tied as he or she does not have the full say in the decision process as long as he or she has a gain from marriage and therefore prefers to not divorce. This may lead dual-earner households to locate close to large metropolitan areas since there is a thicker labor market there (should check the data for evidence). It may therefore be difficult to attract (young) couples to rural locations which could otherwise be a potential solution to the depopulation of those areas. A family might have preferences for living outside the city but chooses not to do so simply because it is harder to find a job for both spouses here compared to in the cities. This could very likely also be the case for single-earner households, but the problem is more severe for the dual-earner ones due to the need of finding two jobs within a fair distance from the residence. Collective models can be used to study the welfare effect on each household member from changes that otherwise only seem to impact one household member. The welfare gain from improving public transport may be higher in collective models because the member that doesn't directly benefit still does so because the household wouldn't have to move in order for the husband to realize the better job further away from the city. Thereby the wife's outside option doesn't deteriorate. 

% \subsubsection{Is the policy of moving academic jobs to rural areas effective in altering couples' moving decisions?}
% Politicians would like to make it more attractive to live outside the biggest cities to avoid the depopulation and stigmatization of the rural areas. In Denmark, the governmental program ``Bedre Balance'' (English: Better Balance) moved around 3,900 public sector jobs to more or less rural areas. These are mainly academic jobs. This may affect the bargaining power within households in which one spouse gets a good job offer from the rural areas because there are now more availble jobs here. Assume the household would in general benefit from the husband taking this job if it also moved to this area where housing is in general cheaper, but that the wife prefers staying in the city. However, since the outside option for the husband has improved relative to the value of staying in marriage if they don't move and let him take the good job, his bargaining power increases as long as there are sufficient gains from marriage such that the wife doesn't want to divorce. This could imply that the household is now willing to move out of the city because the husband's utility matters more when making decisions compared to the time at which the household was formed. A unitary model or a collective model with full commitment would not have been able to make this prediction since it would not let the husbands bargaining power increase. So if the wife really liked staying in the city and both of the spouses preferred marriage to divorce, the household would not see the changes in job prospects in the countryside as a reason to move here. It's not that a unitary model or collective model with commitment cannot predict that the probability of moving out of the urban areas increase when more jobs are available in rural parts of the country, but it would require a relatively larger increase in the job prospects since the probability could not increase because of the household's changing preferences alone. 

% \subsubsection{Are couples less mobile than singles due to the constraint of balancing conflicting interests?}
% Contrary to singles couples have an extra constraint such that they may end up in locations they would otherwise not have considered had they not been married. Also it may be that couples in particular are less mobile (as far as I remember data shows this) due to the dual-earner constraint or it could be because of a selection effect: being in a couple is also a signal for being a ``high moving cost type''. This has consequences for whether politicians should do something about it, i.e. if they should try to loosen the potential constraint. 

% \begin{itemize}
% \item How is mobility affected by better commute system?
% \item 
% \item Mazzaocco (2004) shows that even though individual Euler equations hold this doesn't in general guarantee that aggregate consumption within household does. This thus underlines that groups behave differently than individuals. 
% \item What is the value of having big cities? May be higher for couples than for singles.
% \item If building better public infrastructure => women may become less tied movers in the sense that they don't have to give up their job when their husband gets a good wage offer. This may help reducing the gender wage gap. It may affect also the welfare of the other spouse since he or she doesn't have to transfer as much utility to the spouse that will now commute to the optimal wage location or have to give up his current job because the bargaining would otherwise enforce him to move to another place. So welfare effects may be larger than in individual model.
% \end{itemize}

%In addition, it may bias parameters measuring the utility value of unobserved amenities in the locations if dual-earner households tend to locate in large metropolitan areas because it makes it more likely that both spouses can find a job though they actually really liked the amenities characterizing the countryside regions and only did not choose to live here because of the poorer job prospects. One could 