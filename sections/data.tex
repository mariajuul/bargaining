This section describes the datasets and their variables that I will use in the empirical analysis. Statistics Denmark has provided me access to administrative registers containing information of the entire population of Denmark since 1992-2015 (some registers are not updated for the most recent years yet though). The Technical University of Denmark (DTU) has computed commute times between home and workplaces. Below I go over all these datasets.

\section{Individual and household identifiers}
 The population register BEF holds data on a masked version of the unique and time-constant social security numbers of all individuals with residence in Denmark according to the National Registers of Persons (Folkeregisteret). This is a key variable since it enables me to link information from other registers together. Useful for this paper, the BEF register also contains a household identifier where a household can be either a single person or a couple with or without children. Children living at home belong to their parents' household as long as they live at the same address as one of the parents, are below 25 years old, have not been married or lived in a civil partnership, do not have children themselves and do not make up one part of a cohabiting couple. The household ID is stable over time as long as the person either stays single or the two people in the couple stay as a couple. If a couple splits up, both of them will get a new household ID as long as they do not still live together. The same holds if one of the partners die. In addition to household IDs, BEF thus also tells whether the person lives in a couple or not. A couple can be either a married couple, a civil partnership or cohabiting people with or without children.

\section{Addresses and moves}
From BEF I also get the address of each person in the year. The addresses consist of identifiers for street name, house number, floor and door side and are unique within a municipality, so combined with information on municipality I get the exact address of the individual in a masked version.

The timing of moves can be identified from the date of official change in address. The law requires people to let the municipality know about their change of address no later than 5 days after the move. There is a fine for not complying with these rules and since not registering one's new address means mail is not delivered at the new home, very few people probably do not change their official address very fast. People who move within a year appear with several homes in that year. 

I structure the dataset such that there is only one home per year since the time frame in my model is yearly. Different solutions can be suggested. Statistics Denmark use the residence by January 1st as the home and I decided to use the same approach. In addition to residence information, other personal characteristics from BEF that I use include age, gender, birth region (useful for individuals who were born before 1980 where BEF was started) and lastly mother's and father's social security numbers. From the income registers INDH and INDK I get information on wage income, taxes and total income.

\section{Labor market information}
The population register can be merged on to the Integrated Database for Labour Market Reasearch (IDA). This is a panel of all employments regarding persons living in Denmark since the end of the year since 1980. The database allows me to link individuals and firms and get data on the start year of the employment and number of days employed by the employer who is also occupied with an employer ID. These are based on the start and end dates of the employment that come from the Central Tax Information Sheeet Register (Centrale Oplysningsregister) until 2008 and from eIncome which is also located at the tax authorities. 

Individuals can have several jobs during a year. The register is made up in November of the year and uses the register-based labor force statistics (Registerbaseret Arbejdsstyrkestatistik) to group individuals into either employed wage-earners, employer (A), self-employed (S) or co-working spouse (M). These four groups are mutually exclusive and the difference between A and S is that being a type A means having employees. The category of employed wage-earners can be further divided into main occupation (H), sideline occupation, another November occupation, and most important non-November job. The two latter, however, are only defined from 2004 onwards. The type variable is important when determining which job is the main job to which the individual spends most of the time commuting to. I restrict attention to H and A jobs since the simultaneous home and job moves may be very different for people with S and M jobs since the job in essence can just move with the person. 

It may be, however, that one's H or A job in November is not the job that the individual has had for the major part of the year. In that case the most important non-November job should be considered the job of the year. Since this type is not defined until 2004, I looked at data from 2004-2013 to check how restrictive it would be to define the job in the year as the H or A job no matter what other job categories might be present. I found that for 90\% of the population the most important non-November job had been the main employment for less than half a year and vice versa for H and A type jobs. I therefore decided to use the November employment to define the job of the year. In general, I aim at defining home and job location such that the probability that I model the commute that took place during most of the year is high. In this regard there is a trade-off since I also want consistency in the data over time which is why I do not exploit the non-November employments from the point in time where it was defined. 

In order to be able to model the commute I need information about the location of workplaces. Fortunately, the database allows me to not only link employees and employers but also employees and workplaces (and workplaces and employers). The workplace has an address code attached to it from which I can get province, municipality, parish and traffic zone. In those cases where an employment cannot be assigned to a registered workplace Statistics Denmark will assign a so-called fictitious workplace and the address will be the residence of the individual. This is often the case for people who conduct their work from or near their home or at several different workplaces. The latter concerns, in particular, workplaces for cleaners, insurance and for people working in the social- and healthcare system as for instance a community nurse\footnote{See \url{www.dst.dk/da/TilSalg/Forskningsservice/Dokumentation/hoejkvalitetsvariable/ida-arbejdssteder/lbnr} for a more elaborate explanation.}. Of course this gives rise to problems when calculating the travel time or commute distance for these workers and is something one must have in mind. This can be regarded a measurement error in the workplace variable. The workplace variable is used to get travel times between residential and work location. The travel time estimates will thus tend to be downward biased for people with fictitious workplaces. 

\section{Commute times and distances}
The data on travel times come from The Danish Traffic Model (LTM) developed by researchers at the DTU. In LTM, Denmark as a country has been divided into 907 zones\footnote{There are 4 different zone levels in LTM. This corresponds to level 2, which is rather detailed, and still ensures that the data complies with Statistics Denmark's rules about discretion. See \url{www.landstrafikmodellen.dk} for more information (in Danish).}. The number of trips by use of different transport modes between pairs of zones are estimated in the model. The definition of zones are based on the parish borders which can be linked to the addresses from BEF.

There are different definitions of travel time, namely both by public transportation, car and walk or bike. Travel time by car is given by the sum of free time (minutes in car with free flow, i.e. where the speed equals the allowed speed), congestion time (minutes with congestion, i.e. where the speed is less than the allowed speed), ferry time (minutes sailing by ferry), ferry wait time and pre-departure arrival time that take wait time into account. Travel time by public transport is given by summing waiting time, walk time in connection with shifts between different buses, trains and other public transport modes, walk time to and from first and last stop, respectively, and lastly travel time by other vehicles. In addition, travel time by walk and bike is calculated by LTM for pairs of zones where walk and biking trips actually exist according to the model. 

It is important to note that the LTM has been run for 2002 and 2010 only. Walk and bike time is invariant, but travel time by car and public transportation may change over the years. This is due to for instance new stations/stops being established or existing ones closed, just like construction of a new high way influences travel time. 

% To compute travel times between municipalities\footnote{I am currently waiting for getting access to the merged population and LTM zone data why I cannot use the exact travel times} (or parishes) instead of between zones from LTM I follow the recommendations from researchers at DTU and calculate a weighted average of travel times using the zone pairs within the municipalities in question and weighing travel time in a given zone pair in the municipality by the number of trips made according to LTM. This ensures that when someone is observed to live in municipality A and work in municipality B, I assign the travel time that the average trip takes when accounting for the fact that it is most likely (unconditionally) that this person starts and ends her trip in the zones characterized by most trips. 

% To get the travel time for a given person and year in the dataset, I calculate both travel time by public transport, car, walk and bike for 2002 and 2010. I want one measure of travel time only for each person in the year and use the minimum of all 4 travel times as the representative travel time. This is done both for the 2002 and 2010 versions. The line of thought behind this rule is that I do not observe how people commute. I could, in principle, observe if a person owns a car and thus decide if it is likely that she commutes using the car. However, since cars can be shared within a household and most households own only 1 car, if any, I would have to guess which of the household members used the car. Also, since there can be a huge difference in travel time when using car instead of public transport it would not make much sense to use the mean of the two travel times when calculating travel time by transport. Of course, by taking the minimum of the two travel times, I do underestimate travel time for some people. However, if there is a very huge difference in travel time, from pure intuition it makes sense to assume the fastest mode is used. On the other hand, if the two travel times are not too different, the mistake is not too serious. Another argument for using the minimum of the travel times is that it represents the fastest possible way to get from home to job. I therefore implicitly assume that this is an amenity that people attach to the locations and base their decisions on that rather than also considering whether to buy a car, go by train or bus or choose to walk or bike.

LTM also provides a measure of the distance between zones. In addition, a work distance variable is available and calculated based on actual work and home addresses, but currently only available for the entire population from 2000-2008. As above the actual work address is however unobserved for those who do not have a registered address because they do not have a formal workplace but rather drive around the country to sell e.g. insurance products. There is a choice between using work distance vs travel time to measure the burden associated with the commute. The arguments for using travel time is that two jobs located in the same distance from some home location may give rise to very different travel times dependent on congestion and public transport availability. On the other hand, travel times are only available for 2002 and 2010. For now, I use work distance.  %on commute tax deductions from the tax authorities. These tax deductions are observed in the registers and comply to every employed person who lives more than 12 km away from his workplace. Until a commute distance of 100 km, there is a base deduction rate that varies from year to year. Hereafter, the tax deduction rates is half of the base rate\footnote{\textcolor{red}{}See [cite appendix] and \citet{Munk-Nielsen2015} for details.}. Using these rules, a measure of the work distance can be obtained. Note, however, that it is not the exact work distance for each individual. For those who commute less than 24 km per day (to and from work), the tax deductions are 0 and work distance set to 0 as well. Moreover, I observe total deductions which are a computed as the deductions per day times number of days worked. The latter is unobserved why it has been set to 225 days in all years for all persons. This corresponds closely to the the official number of work days in all years. however, the correlation increases markedly to \textcolor{red}{0.89}. This is likely an artifact of the above-mentioned problem with people who do not have a registered workplace as these people's workplace is set to the home address. If they actually do not work in same municipality as they live but rather drive around the country to sell insurance products for instance, their commute tax deductions and thus the work distance variable will reflect this. 

\section{Home ownership and home characteristics}
From the register of home characteristics, BOL, I observe characteristics of all dwellings in Denmark such as number of rooms, whehter the property has toilet and kitchen, living area, basement area, ground area, year of construction, number of buildings on the lot, type of home (e.g. townhouse, apartment, summer cottage), whether the dwelling is listed and its address. Additionally, EJSA tells what the transaction price of the house was at the time of a sale as well as the public evaluation of the home. One does not, of course, have to own a home. This information comes from the register EJER that contains information about who owns a given building or part of a building. 

\section{Local amenities}
Regions are characterized by different amenities and I use data on air quality, crime level and school quality in this study. 

Data on air quality come from the Danish Center for Environment and Energy who model the air pollution in Denmark. These pollutants include, most importantly, $NO_x,O_3$, $PM_{2.5},PM_{1.0}$ and $CO$, all computed in $\mu g/m^3$ for each $1\times 1$ km square cell of Denmark for 1979-2015 on a monthly basis. To allign this dataset with the other registers, I have computed yearly averages for each pollutant and square cell. 
%add more about how computed. 

Furthermore, Statistics Denmark provide information on number of victims by type of crime by year and $0.1\times 0.1$ km, $1\times 1$ km and $10\times 10$ km square cells of Denmark. There is a very high level of discretion for this register, though, which means that for the 0.1 definition, the number of victims has been anonymized by setting it to missing for many of the cells. A cell is a square km-by-year-crime type combination. However, it is possible to aggregate to e.g. municipal or parish level and thereby get useful information. The types of crime are violent crime, sexual crime and property crime. More detailed crime types are available, e.g. distinguishing between burglary and theft in property crime but then there are much more cells that are anonymized and hence less useful for the empirical analysis.

Data on school quality come from the Ministry of Education and contain GPA by school, school year and subject (available for 2005-2015). Identifiers of the schools can be merged on to Statistic Denmark's registers for school districts which again can be merged on the residential addresses of the individuals. Thereby it is possible to assign a school to each home. Each home belongs to a school district which contains a school. In 2005, the free school choice reform was implemented implying that the parents in principle can choose from any other school. This requires, however, that there is an available spot for the extra pupil in the school. There is no guarantee for this as is the case with the school in the home's school district though.
