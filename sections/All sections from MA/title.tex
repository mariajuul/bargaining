
\begin{titlepage}
\Large
\begin{spacing}{1}
\title{\textsc{A dynamic structural approach to home and job location decisions}}
\author{
\textsc{Maria Juul Hansen} 
\thanks{University of Copenhagen, Department of Economics {\tt mjh@econ.ku.dk}} \\
\textsc{\small University of Copenhagen} 

\date{\textsc{October 2016}}

\maketitle
\thispagestyle{empty}

\begin{abstract}
This paper presents a dynamic structural model of the joint decisions of in which location to reside, whether aiming for being employed and, if so, where to work. It is a finite horizon equilibrium model where housing prices are modelled as an equilibrium object reflecting the expected supply and demand of homes in the regions. The model is dynamic in the sense that the household decides at each age whether it wants to move to another place. The household makes this decision by taking into account and forming expectations about location-relevant and time-varying characteristics. At the same time the decision of work location is modelled. The individual is subject to region- and individual-specific unemployment risk, but regions also differ in terms of income potential and commute costs dependent on the home region. In the current version of the paper I solve a simplified version of the model for two home and work locations and show that commute costs are important for explaining differences in housing prices across regions.
\noindent

\bigskip

\noindent \textsc{Keywords:} Structural estimation, Dynamic Equilibrium models, Transaction cost, Urban Economics.

\end{abstract}


\end{spacing}
\end{titlepage}

