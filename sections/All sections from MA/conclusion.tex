Motivated by the increasing urbanization trends seen worldwide and the focus of the Danish government to avoid a too rapid depopulation of rural areas, this paper looked into the determinants of location choices in terms of residence and workplace. The literature on this topic has a long history but with the general framework of focusing on either the residential or the job location choice. This is a harsh assumption to make when it is not clear whether people move jobs because they would like to live somewhere else or vice versa. This paper therefore took a more general approach by considering these decisions to be simultaneous. I started by doing descriptive analyses of the distribution of people across the country by showing that there is a tendency to cluster around areas with more jobs and educational institutions. Also, the data showed evidence of sorting in terms of education, i.e. that highly educated cluster in certain areas. Moreover, I delved more into the commuting patterns. Individuals in Denmark tend to have a rather short commute time, especially in the urban areas. Most home moves are associated with not moving job in the year or the year before or after, but a significant fraction also change job either the year before or in the same year as moving. I used the descriptive evidence to motivate a dynamic structural model of these location decisions. Contrary to the major part of the literature, I allowed housing prices to be endogenous and thus respond to the location choices made by the individuals. I solved a simplified version of the model including the equilibrium for different scenarios. First, using data on commuting costs from provinces of Denmark, I showed how different travel times to and from regions of Denmark are consistent with housing prices being higher in the Copenhagen area where there is in general shorter commute distances due to many jobs being located here. Additionally, the model proved to predict sorting patterns in line with those observed in the data. Namely that highly educated have a higher tendency to live and work in urban areas than do people with no or low education. Lastly, I ran a counterfactual simulation where travel time was reduced by 60\% for provinces outside of Copenhagen. There was evidence of people reacting to this by being more willing to live outside the big cities. In the current version of the model this was reflected in the upward pressure on housing prices in these more remote locations. Even though the model can produce results that are consistent with patterns observed in data, one should keep in mind that I used an extremely simplified version of the ultimate model that I aim at implementing at a later stage. Just like I have not estimated it yet, but save that for future work. In particular, this future work includes finding a solution to the curse of dimensionality which has been identified in the full version of the model. In the model implemented in this thesis, I assumed that individuals were guaranteed employment if they wanted to work no matter the location. This is unrealistic, why the full model accounts for the fact that people do not necessarily have job offers everywhere. This means individuals must draw a choice set every period and that is exactly what causes the curse of dimensionality. Next step will be to restrict the number of job offers each period to 1, but without saying exactly where it comes from and explore how that helps reducing the computational burden. Additionally, I plan to integrate an outside option such that the model can be used to predict how urban vs rural locations will grow in terms of population over time. So far the prices of housing are the only adjustment mechanism when demand for living in specific areas increase compared to the supply. This is not a very bad assumption in the short run, but since interest also lies in explaining why some areas grow in size and others do not, the outside option will be an important extension. 
% This paper investigated the predictions of a dynamic discrete location choice model in a finite horizon life cycle setting. The choices to make were in which out of 2 regions the individual wanted to live and work, respectively. The model was a simplified first version that included two state variables, namely age and previous home location. The latter mattered because there were switching costs involved when moving residence. By use of backwards induction, I solved the model conditional on house prices in each residential location. I extended the framework to allow for endogenous prices on the housing market, determined by excess demand. I then showed that predictions of the model would differ dependent on which framework I used. Looking into different counterfactuals, I found that higher switching costs lowered the probability of moving, a higher wage potential in one region made it more likely to work here, but it did not affect equilibrium house prices until also commuting costs were added. I showed that the CCPs of the different alternatives were highly dependent on age. Among other things, I found that the probability of moving was lower for old people. Something which is consistent with empirics. Additionally, I investigated how equilibrium house prices would respond when not only the wage level, but also wage growth differed between regions. The result was that the house prices of the high wage growth region would increase steadily until around 10 years after the introduction of different wage paths. After that period of time, relative prices stagnated, but were considerably higher in the high wage growth region. Again, this is consistent with empirics. Hence, even though this is an extremely simple model, already a few empirical patterns can be matched. For future work, I plan to extend the model both in terms of the state space and heterogeneity in the utility function.