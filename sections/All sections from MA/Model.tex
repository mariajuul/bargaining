The model presented in this section is the one I aim at estimating at a later point\footnote{The model from which I show simulations in \autoref{sec:simulations} is a simpler version of the model presented here.}. It is an equilibrium finite horizon life cycle model of location decisions at the household level, i.e. I do not take intra-household decisions and bargaining into account. I let the household representative be the decision maker when not dealing with a single person household. It is a version of the location and occupational choice model from \citet{Buchinsky2014}, but where the occupational choice between white-collar, blue-collar and unemployment is boiled down to distinguishing between employment and unemployment. The latter may be involuntary via firings. I therefore restrict attention to the overall labor market participation decision and residential and job location choices. Regarding employment, work regions differ in terms of wage potential. Consequently, the model is able to capture that some regions for that reason are more attractive to work in. The model of the current paper extends complexity in other dimensions compared to \citet{Buchinsky2014}, which is the most complex model within the subject to date, namely by endogenizing house prices. As in most of the literature, where \citet{Oswald2015} is an exception, I do not distinguish between home owners and renters. Hence, I impose that every household must buy a home if deciding to move to a new region and must sell their current house. This means all households make the high upfront investment when moving residence and also expect to be able to sell the house at some price later on if they want to, thereby expecting a potential capital gain from moving between regions. However, I do not attempt to model households' expectations about housing price increases. This means I assume everyone expects housing prices to stay constant over time. This does not mean housing prices in fact do not change as time goes by, for instance due to amenities changing in the regions, but it does imply that the decision maker does not buy a home with the expectation of making a good deal in the sense that he buys when prices are low and will sell when prices are high in the future. I make this choice in order to avoid carrying amenities and housing prices as state variables that should be used to form the expectations about these for coming years. That would complicate the solution of the model a lot. 

I let job locations and wages be exogenous to increase chances that the model is tractable for estimation. As a consequence, I do not consider the effect that migration patterns have on relative wages across regions. Whether the tendency to migrate to the cities have a significant effect on the relative wages depend, among other factors, on the substitution between the original inhabitants of the cities and the newcomers. If the substitution is high, we should expect to see falling relative wages in the urban relative to rural areas. 
%NB: find out what empirical evidence shows 

Let $i$ denote the individual, $rh\in D^{rh}=\{1,2,...,\bar{rh}\}$ the universe of residential locations and likewise $rw \in D^{rw}=\{\emptyset\} \cup \{1,2,...,\bar{rw}\}$ the job locations, where $\emptyset$ denotes unemployment. Unemployment is seen as a both an involuntary and a voluntary choice. Hence, individuals are both allowed to choose themselves to be unemployed and they also risk being unemployed despite their wish to work if they cannot find a job. I will elaborate on this matter below. For the set of regions $\bar{rh}=\bar{rw}$, i.e. there are no regions where the individuals cannot work or live. Regions differ in terms of a number of characteristics to be spelled out later.  $t$ indexes the individual's age and $d_{it}\in D^{rh}\times D^{rw}\equiv D$ is the choice made by individual $i$ of age $t$. Decisions are made in the beginning of the period and the consequences take place immediately after in the same period. 
 %Regions are heterogeneous in several dimensions, among others in terms of wage potential, i.e. some regions have a higher general wage level than others. This could reflect things such as differences in living costs, given that most people live in the same region as where they live, overall different productivity levels due to diverse educational backgrounds across the regions, differences in the products and services sold, etc. This is taken into account in the model by allowing for region-specific base wage levels. On top of the base wage comes a return to age (as a proxy for experience)


\section{Individual optimization}
Each period until (deterministic) death at time $T$, the individual chooses where to live, whether and if so where to work. At age $t$ he therefore decides on the entire sequence of decisions until $T$. In each period he gets the flow utility $u^{d_{it}}(x_{it})$ and the expected value, $EV_{t+1}$, of future periods from making choice $d_{it}$ given his state $x_{it}=(rh_{it-1},rw_{it-1},kids_{it},$ $\mathbb{I}({couple_{it}}),\mathbb{I}{(s_{it}>\bar{s})},t,\eta,j,m_t)$ to be defined in the following: $rh_{t-1}$ and $rw_{t-1}$ are the home and job locations, respectively, at age $t-1$. They can take the same values as $rh$ and $rw$. $kids_{it}\in \{0,1,2,3+ \}$ is the number of kids living at home. I do not distinguish between having 3 or more kids in the household.  $\mathbb{I}({couple_{it}})$ is an indicator equal to 1 if the individual lives in a couple, $\mathbb{I}{(s_{it}>\bar{s})}$ a dummy for having schooling level above some threshold $\bar{s}$. $t$ is age, $\eta\in \{0,1\}$ an unobserved income type (``high'' vs ``low''), and likewise is $j\in\{0,1\}$ an unobserved moving cost type (``stayer'' vs ``mover'' type). Both of these are assumed constant over time like other preference parameters are. Lastly, $m_t$ is a macro indicator which takes the value 1 if the economy is in a boom and 0 otherwise. On top of that comes the alternative-specific taste shock $\epsilon_{it}^{d_{it}}$. The motivation for including these state variables will be elaborated in the coming sections. 

\subsection{Job offers}
Individuals' choice sets vary over individuals and time. The choice sets differ in terms of which $rw$ can be chosen as work regions. This is done to restrict work location choices to only include those regions where the individual in fact has a pending job offer, though $rw=\emptyset$ is always in the choice set. Which $rw$ are part of the choice set depend on individual characteristics and in particular whether he had a job in the previous period or not. If he did, that work region is in the choice set if he does not get ``dismissed from the current work region''. This can come about in two ways: either the person gets fired from his current job but gets a job offer in that same region or he does not get fired from his job, which corresponds to receiving an offer to continue the contract.  To simplify matters, I let the involuntary dismissal probability denoted by $P^{dis}$ be independent of the work regions. This is in line with \citet{Buchinsky2014}. On the other hand I do allow $P^{dis}$ to depend on the individual unobserved income type $\eta$, though I do not plan to estimate $\eta$ itself, but rather the share of each type in the population. Assuming $P^{dis}$ can be described by the Logit formula as in \citet{Buchinsky2014}, it is given by
\begin{align}
P^{dis}_{\eta} = \frac{\exp{(\alpha_{\eta})}}{1+\exp{(\alpha_{\eta})}}, \eta\in\{0,1\}.
\label{eq:pdis}
\end{align}  
Since the agent belongs to the same $\eta$ throughout his life \eqref{eq:pdis} implies that the probability of being laid off from one's job is independent of age (and other state variables).

As mentioned above, the individual may get job offers from other regions where he does not currently work. Denote this probability $P^{job,rw}$, where $rw$ indicates that the job arrival probability depends on the region considered. Let this be be given by 

\begin{align}
P^{job,rw}_{it}=\frac{\exp{(A^{rw}_{it})}}{1+\exp{(A^{rw}_{it})}},
\label{eq:pjob}
\end{align}

where
\begin{align*}
A^{rw}_{it}=\alpha_0^{rw}+\alpha_1^{rw}\mathbb{I}{(s_{it}>\bar{s})}+\alpha_2 \mathbb{I}{(rw_{t-1}\neq \emptyset)}+\alpha_3\cdot t + \alpha_4 \cdot t^2 + \alpha_5 \cdot m_t+\alpha_6 \cdot  \eta.
\end{align*}

As seen from \eqref{eq:pjob} I let the job offer probabilities differ across regions in terms of a constant. Also, I allow highly educated people to get more or less job offers than low educated, all else equal. To recognize the fact that some regions have a higher concentration of high-skilled jobs than others, cf. \autoref{sec:data}, the effect from education is allowed to differ across regions. This will help the model explain why groups of equally educated people to some extent tend to cluster in specific regions. Despite no age profile in the dismissal probabilities, I do include that in \eqref{eq:pjob}. Conditional on having a job I thus assume that employers are not more likely to fire the employee just because he is older or younger. However, there may be an effect on the job arrivals from other regions. Since the state of the economy likely highly affects the number of jobs available, the macro dummy is added in addition to the unobserved income types $\eta$. The latter captures that some individuals are just more lucky, have a wider professional network, are very talented compared to those people that seem similar on observables, and other unobserved mechanisms that can explain why some people are more likely to get job offers. 
 
The model is such that when the individual makes his decision on $(rh,rw)$ he knows that this is with an employment probability of 1 as long as $rw\neq\emptyset$. The scenario where one chooses to work in $rw$ but happens to not get a job thus does not exist. This means I do not model the search process as such but consider that an underlying, unobserved process going on beneath the surface of the model. Hence, another way to interpret $P^{job,rw}$ is that it also captures the search intensity of the individual as a function of his characteristics besides the probability of actually getting a job offer. These are of course somehow interrelated but for the model it is not important whether many job offers arrive due to a high search intensity or because the individual just exogenously receives many offers of employment because employers tend to find that person attractive since I do not model the search intensity, at least not in the current version. The very important part of this model is that it allows individuals to make the moving decisions \textit{conditional} on knowledge about employment options. This does not restrict individuals from moving to some place $rh$ without having a job nearby, but it does mean that no one moves and gets surprised that he could not find a job in the vicinity of the new home. If that is the case it was a \textit{choice} made by him. It may a choice of unemployment for the current period but with the expectation that job offers will arrive in future periods though. Contrary to the current period outcomes, unfulfilled expectations about the future are by definition a surprise. 

Additionally, the model encompasses the realistic situation that most people would be unlikely to quit a good job before knowing if they have another one waiting for them. Since I do not restrict the number of job offers an individual can receive, it is possible to both not be laid off from one's current job (i.e. get an implicit job offer from here) and to have job offers from all the other regions\footnote{This is something that is doable in terms of of solving the model when there are not ``too many'' regions in the choice set and the time horizon is short. When the choice set grows, allowing for an unrestricted number of job offers becomes infeasible.}. This resembles the approach in \citet{Buchinsky2014} but is otherwise different from what is done in most of the literature so far. Another much more computationally light approach would be to assume agents decide on their $rw$ based only on \textit{expectations} of where job offers are pending both for the current and future periods. This is in line with, among others, \citet{KennanWalker2011}, \citet{Oswald2015} and \citet{Ransom2016}, where employment options are experience goods implying one has to move to the region in order to get a sense of whether or not he will be employed here. The way job offers enter when applying this approach is by letting individuals form expectations about the job offer arrival rate in either region. This set-up is easier to handle in terms of solving and estimating the model (but not necessarily for fitting it to data) since the econometrician would not have to model how job offers arrive in order to solve the model and would not have to care about how this should be dealt with in estimation when job offers are not observed in the data.

A relevant discussion to take at this point is therefore how crucial it is to model actual job offer arrivals. The thing is that it is very risky for most people to move to another place far away from one's current job if not knowing whether he has a job close to the new home. It is therefore very likely that people do have a very good feeling of whether moving means moving to unemployment, another job or staying in current job before making their moving decisions. An alternative model where actual job offers are unknown would give rise to potentially serious error when predicting people's behavior from their state variables. In fact, the model would have a hard time explaining the behavior of an individual who is observed to have a job at $t-1$, moves to a new residential location at $t$ far from previous home and work unless there is a \textit{very}, almost certain chance that he will get a job in a manageable distance from the new home. Otherwise the individual is giving up a potentially good job with short commute from the old home location and risking either getting unemployed (if he quits his old job) or gets a very long commute. To fit such a model, marginal utility of income must be very, probably unrealistically, low or commute costs must be so. For reasonable parameter values there is therefore a high probability that moving home seems like a suboptimal decision. The model would infer that he must have got a very good taste shock for decisions involving moving to the new home location and shocks of that size happen only with very low probability. 

One way to try to get around this matter is to assume that there is an employment probability of 1 in all regions. The argument is that it is always possible to get \textit{some} job even though it may not be one's favorite job. However, such a model would have to distinguish between different employers within the same region in order to recognize that there are good jobs and bad jobs in each location. Otherwise employer characteristics would be region-specific and one region could not be characterized by both good and bad jobs according to the model.  This would require e.g. employer-employee-specific match components, but this would just introduce another unobserved variable instead of the job offers. The econometrician would thus still have to come up with a feasible strategy for how this should be handled when estimating the model and the gains are not immediately obvious. Even if abstracting from employer-specific characteristics within a region the model would predict too many job moves. Every time there is a better job, the individual should go for it since he knows he will get it. However, people do not constantly change jobs. A way to try to solve this in a reduced-form way is to impose job search costs. This means the model knows it is not completely costless to try to change job since the time the individual spends on job searching comes at an opportunity cost. The implication of introducing job searching costs is that the model is less inclined to predict that the optimal choice is to choose another $rw$. First of all, even though unemployment risk would not be directly modeled by this approach it should to some extent be captured by just a constant job search cost: by giving up your current job to search in another region, you risk being unemployed. However, in reality, it would be relatively seldom to observe someone who voluntarily quits his job before getting another one given he actually wants to still work. There is on-the-job-search which that model does not allow for. In principle, it would be possible to include on-the-job search by extending the decision to include multiple work regions, e.g. allowing the individual to choose his current work region plus another one. That would make the model solution and estimation even more difficult though. The struggle is that the individual can only end up working in one of the regions at a time. Adding the choice of staying in your work region would mean there should be some (stochastic) process afterwards that determines in which region he ends up working. This is essentially the same set-up as that of the current description where the individual knows where he has job offers, because the outcome process should be determined somehow and the one to determine this is the agent. He will make his decision based on where the most suitable job for him is located. Moreover, this alternative model assumes agents \textit{choose} to be unemployed. This is an important assumption since in the data many of those people who are observed to be unemployed probably did not choose to be so. Nevertheless, in most regions the individual would always be able to get \textit{some} job if he lowered his reservation wage. With this in mind it may not be as harsh an assumption to make that unemployment is voluntary. In that sense being unemployed simply means being a non-labor market participant. You do not search for a job if you do not want one, and you do not want one if you are unemployed. Even though this simpler model could be implemented I find that the assumptions about how job search and job outcome are determined are too indirect and a too reduced-form approach. The costs in terms of understanding what is actually going on and how to interpret the parameters of the model exceed the extra cost it is to model the labor market more realistically with job offers arriving before the decision is made. Even though that version of the model cannot be fully solved by now, I am convinced it is worth continuing to work along that direction.

%Also since the time frame is one year which means that it is quite likely to get \textit{some} job during a year if that is what you really like to. 

%The ideal would be to allow agents to know where they had been offered jobs and then model the choice of where to live and where to take a job. However, I do not observe people's job offers in the data. I could calculate $EV$ of being in a certain state conditional on a vector describing in which of the $|rw|$ regions the agent had jobs. I would then calculate the alternative-specific value function $v^{rh,rw}_{|job^{rw}=1}$ and $v^{rh,rw}_{|job^{rw}=0}$ where $job^{rw}=1$ means the person has a job offer in $rw$. If future job offers are not a function of today's employment outcome and region of work, I would then calculate $EV(x)_{|job}$, where $job$ is a $rw\times 1$ vector of dummies for having a job in each of the $rw$s. This vector can take $2^{|rw|}$ values (given you do not restrict the number of job offers an individual can get per time period) and I would have to compute $EV$ for all combinations. This would provide me with the model solution conditional on knowing the job offers. However, since these are unobserved, I must integrate over the joint distribution of $job$ to get a feasible model solution. For each state $x \setminus job$ I could e.g. draw $M \rightarrow \infty$ times from the $job$ distribution, evaluate $EV(x)_{|job}$ and take the mean over the $M$ versions of these. But this expectation is very high-dimensional and it would be infeasible to do repeatedly in full solution types of estimators where the model is solved and the integral thus computed for each guess of the parameter. The number of combinations grows exponentially in the choice set and already when working with 20 work regions it would amount to more than $1,000,000$ different combinations that I would have to integrate over. Also, I do not know the distribution of $job$. That would be something I should estimate from data and since I do not have information on people's job offers that would be yet another challenge to identify this. In principle, one could get information on the aggregate number of job offers in each region, potentially by some category, but there is no corresponding data on the number of applicants for each job available, why this is not directly applicable in this setting. Given that the arrival probability of job offers in the future very likely depends on today's employment status and region one would also have to condition on the arrival of future job offers. Even if only today's employment status affects tomorrow's job offer probabilities, the solution to the model gets just more complicated. \\

%Another much more computationally light approach would be to assume agents decide on their $rw$ based only on \textit{expectations} of where job offers are pending both for the current and future periods. This is in line with \citet{KennanWalker2011}, \citet{Oswald2015} and \citet{Ransom2016} and easier to handle in terms of estimating the model (but not necessarily for fitting the model) since the econometrician would not have to integrate $EV$ with respect to the joint distribution of job offers. Instead, the agent just maximizes the expected $EV$ in terms of job offers and one does not have to know anything about actual job offers in order to obtain the model solution. One could let the employment probabilities depend on his state $x$ incl. age, distance between $rw_{t-1}$ and $rw$ to proxy for the idea that the hassle of searching gets worse if you have to search for a new job far away from your current job and professional network. First off, this is of course a quite crude approximation of reality, secondly there is still an identification problem of the probabilities with which the individual gets employed in each region conditional on his search. The statistic the agent should respond to is the probability of getting a job in $rw$ given he searched for one. This is defined as the number of people characterized by state $x$ that got a job in region $rw$ compared to the total number of people in state $x$ that \textit{searched} for a job in $rw$. This is practically unidentified. Since there is no good solution to the latter I decide to go with the model where there is no involuntary unemployment (there is no involuntary unemployment in the first model alternative described either given that the agent does receive a positive number of job offers) and where unemployment risk is completely disregarded. \\

\subsection{Value functions}
Applying this approach means the individual observes $(x_{it},\epsilon_{it}, D_{it})$, where $D_{it}$ is the individual- and time-specific choice set, before making his optimal sequence of decisions. He does this by maximizing the expected sum of discounted utility as formulated in the Bellman equation 
\begin{align}
V_t^D(x_{it},\epsilon_{it})=  \max_{\{d_{it}\}_{t}^{T}\in D_{it}} \Big\{\left[u^{d_{it}}(x_{it})+ \epsilon_{it}^{d_{it}}+\delta EV_{t+1}(x_{it}) \right]\Big\}. 
 \numberthis \label{eq:V}
\end{align}
$V_t^D(\cdot)$ is the value function for age $t$ under $D_{it}$ (value functions differ across choice sets since having more choice alternatives to choose between cannot make anyone worse off). The number of choice sets is $2^{|rw|}$ and thus grows exponentially in the number of regions\footnote{All home regions are always in the choice sets. These therefore only differ in terms of the work regions.}. $EV_{t+1}(\cdot)$ is shorthand notation for $\mathbb{E}_{\epsilon_{t+1},x_{t+1},D_{t+1}|x_t,\epsilon_t,d_t}[V_{t+1}(\cdot)]$, i.e. the expectation of next period's value function conditional on current period's state and decision. The expectation is taken over $\epsilon$, the uncertain, time-varying variables in $x$ and the different choice sets $D_{t+1}\in D$. The latter is done because even though the agent knows in which regions he has a job offer today he can only guess about whether and where job offers will arrive in coming periods. To form this expectation he uses the probabilities given by \eqref{eq:pdis} and \eqref{eq:pjob}. The remaining transition matrices will be spelled out later. Let the choice-specific value function be defined as 
\begin{align}
v_t^{d_{it}}(x_{it})= u^{d_{it}}(x_{it})+\delta EV_{t+1}(x_{it}), \label{eq:v} 
\end{align}
Using \eqref{eq:v} allows me to follow \citet{Rust1987} and rewrite \eqref{eq:V} into
\begin{align*}
V_t^D(x_{it},\epsilon_{it})&=\max_{\{d_{it}\}_{t}^{T}\in D_{it}} \{v_t^{d_{it}}(x_{it}) +\epsilon_{it}^{d_{it}}\}
\end{align*}
$v_t^{d_{it}}(\cdot)$ does not depend on the taste-specific shock $\epsilon_{it}^{d_{it}}$, because they are assumed independently and identically distributed over time, individuals and alternatives such that they do not enter $EV_{t+1}(\cdot)$. This stochastic shock is observed by the individual just before making the decision for the current period but unknown to the econometrician like future shocks are unobserved to the individual himself. $\delta$ is the discount factor, potentially a function of age to account for increased mortality risk the older one gets. I assume that individuals know the distributions of both the state variables and the taste shocks such that they can calculate the expectation of these for future periods. The additive separability between deterministic utility and the shock as assumed in \eqref{eq:V} is common in the literature and introduced by \citet{Rust1987}. This is done together with the assumption of conditional independence between $\epsilon$ and $x$ and that $\epsilon$ is independently and identically distributed across alternatives, individuals and time. Imposing this structure will help for identification, which is generally difficult in dynamic discrete choice models. The probability of each choice set for future periods depends on the individual's characteristics and employment status in the next period. To form the expectation about the choice set, the individual therefore must form expectations about the evolution of the state. This is summarized by \eqref{eq:condi}
\begin{align}
P(x_{t+1},\epsilon_{t+1},D_{t+1}|d_t,x_t,\epsilon_t))=g(\epsilon_{t+1}|x_{t+1})f(x_{t+1}|d_t,x_t)q(D_{t+1}|x_{t+1}),
\label{eq:condi}
\end{align}
where $g(\cdot)$, $f(\cdot)$ and $q(\cdot)$ are the p.d.fs of $\epsilon$, $x$ and $D$, respectively. The assumption of conditional independence implies that once the choice $d_t$ is made, one cannot use $x_t$ to predict $\epsilon_{t+1}$ and vice versa. I.e. if an individual is observed to live and work in $(rh,rw)=d_{it}$ the taste shock is for example not informative on whether he will have more kids in future years when we also know $x_{it}$. As it has become standard practice in dynamic discrete choice models dating back to \citet{McFadden1974} I assume $\epsilon$ is i.i.d. Extreme Value Type I distributed across alternatives, individuals and time with common scale parameter $\sigma(x_{t+1})$ (henceforth suppressing $x_{t+1}$). Using these distributional assumptions, I take the conditional expectation of $V^D_{t+1}$ with respect to $\epsilon'$ and exploit the neat future of the Extreme Value distribution that the expectation has a closed form, see \citet{Rust1987}:
\begin{alignat*}{3}
&\int_{\epsilon'}{ V_{t+1}^D (x_{it+1},\epsilon_{it+1})g(d\epsilon'|x')} &&= \int_{\epsilon'}{ \max_{\{d_{it+1}\}_{t+1}^{T}\in D_{it+1}} } \Big\{v_{t+1}^{d_{it+1}}(x_{it+1}) +\epsilon_{it+1}^{d_{it+1}}\} \Big\}g(d\epsilon') \\
& &&=\sigma \log {\left( \sum_{d_{it+1} \in D} { \exp[v_{t+1}^{d_{it+1}}(x_{it+1})/ \sigma]  } \right)} \\
& &&\equiv\phi^D(x_{it+1}). \numberthis
\label{eq:EVeps}
\end{alignat*}
Using \eqref{eq:EVeps} and the distribution function for $x'$ and $D'$, I can now rewrite \eqref{eq:v} as
\begin{align}
v_t^{d_{it}}(x_{it})&=\Big(u^{d_{it}}(x_{it}) +\delta \int_{x'} \sum_{D'}{\phi^D(x_{it+1})\cdot q(D'|x',d)f(dx'|x,d)}\Big).  \label{eq:vphi}
%v_t^{d_{it}}(x_{it})&=\Big(u^{d_{it}}(x_{it}) +\delta \sum_{D'} \int_x[{\phi^D(x_{it+1})\cdot f(dx'|(x,d)]q(D'|x,d)}\Big).  \label{eq:vphi}
\end{align}

\subsection{Solution procedure}
To find the optimal solution path $\{d_{i}\}_{t}^{T}\in D_{it}$ for the dynamic discrete choice problem originally written as \eqref{eq:V}, we solve for the alternative-specific value functions from \eqref{eq:vphi} by use of value function iterations. At age $t$ the solution procedure is therefore to start solving for period $T$ for each possible value of $x_T$ and for each choice set $D_T\in D$ to compute \eqref{eq:vphi} under the restriction that $\phi^D(x_{T+1})=0$. This means agents do not care what happens after they are dead, so altruism towards kids and future generations hence are non-existing. This makes the solution to $v$ very easy at age $T$ since it is simply given by  $v_t^{d_{it}}(x_{it})= u^{d_{it}}(x_{it})$. To form $\phi^D(x_T)$ we take the expectation over $\epsilon_T$ according to \eqref{eq:EVeps}. This provides us with the expected value of ending up in state $x_T$ for the choice set $D_T$ by the end of life. Next, we solve the maximization problem at age $T-1$. We already know $\phi^D(x_T)$ from the previous iteration and can thus calculate the value of each choice alternative by use of \eqref{eq:vphi}. Again, we must take the expectation over $\epsilon_{T-1}$ to get $\phi^D(x_{T-1})$. Additionally, we take the expectation of $\phi^D(x_{T-1})$ with respect to the choice sets and over state variables for period $T$. This procedure continues until we reach the current age $t$. At this point the individual is assumed to make his decision $d_{t}$ taking into account how that affects $x_{t+1}$ and hence the optimal decision $d_{t+1}$, the consequence of that on $x_{t+2}$, etc. 

However, since we as econometricians do not observe $\epsilon_t$, but the solution just described is a function of this, we must integrate over its distribution. We therefore cannot tell what the exact optimal solution at time $t$ is for state $x_t$. Instead we find the conditional choice probability (henceforth CCP) which describes the probability that a given choice $d_t$ is optimal conditional on state $x_t$. Given the distributional assumption on $\epsilon$ the CCP is
\begin{align}
CCP(d_{it}|x_{it})=\frac{\exp[v_t^{d_{it}}(x_{it})/ \sigma]}{\sum_{d_{it} \in D} { \exp[v_t^{d_{it}}(x_{it})/ \sigma]  }}
\end{align}
and we use these to describe the model solution. Until now in this outline I have assume $D_{it},\eta,j$ are known to the econometrician, but this will be relaxed when the model is estimated in future work.

\section{Utility specification}
When an individual of age $t$ who by the end of age $t-1$ lived in $rh_{t-1}\in D^{rh}$ and worked in $rw_{t-1}$ makes decision $d_{it}$, he gets instantaneous deterministic utility
\begin{alignat*}{3}
&u^{d_{it}}(x_{it})=&&\kappa^{mu}(inc_{it},m_t){inc_{it}^{rw}}(x_{it})+taste_{it}^{rh}-hcost_{it}^{rh}-swcost_{it}^{rh,rh_{t-1}}(x_{it}) \\
& &&-comcost_{it}^{rh,rw}. \label{eq:u} \numberthis 
\end{alignat*}
This very much resembles the specification of utility in \citet{Buchinsky2014}. $inc_{it}^{rw}$ is total income of the individual when he works in region $rw$ (including unemployment $rw=\emptyset$) and $taste_{it}^{rh}$ aims at controlling for amenities of residential location $rh$. $hcost_{it}^{rh}$ is the cost of living in region $rh$ and $swcost_{it}^{rh,rh_{t-1}}$ the switching cost when moving from region $rh_{t-1}$ to $rh$. Lastly, $comcost_{it}^{rh,rw}$ is commuting costs between residence $rh$ and work $rw$. 

In the following I describe the separate components of \eqref{eq:u} starting with $\kappa_{mu}(inc,m)$. This is the marginal utility of money. It is assumed to depend on both the income level and the macro state to allow high-income people to have a lower marginal utility of money and to let macro state proxy for the individual's optimism, which may affect his willingness to pay for certain products, incl. the costs associated with moving. This is done to implicitly account for budget constraints in the model, which are not imposed explicitly. An individual with a high marginal utility of money will be less inclined to pay the costs of moving, all else equal, just like a person who is close to not satisfying his borrowing constraint is. This follows \citet{GillinghamEtAl2015}. Following them I let
\begin{align*}
\kappa^{mu}(inc,m,\theta^{mu},\kappa^{mu}_b)=\kappa^{mu}_b\times\frac{1}{1+(\kappa_0^{mu}+\kappa_1^{mu}inc+\kappa_2^{mu}inc^2+\kappa_3^{mu}m_t)},
\end{align*}
where $\theta^{mu}=(\kappa_0^{mu},\kappa_1^{mu},\kappa_2^{mu},\kappa_3^{mu})$ and $\kappa^{mu}_b$ is baseline marginal utility of money, i.e. the highest obtainable marginal utility of money, which is attained by individuals who have an income of zero in a bust year. The income process itself is assumed deterministic for now\footnote{I will add stochasticity in terms of e.g. an AR(1) component later.} and given by 
\begin{align}
inc_{it}^{rw} = \begin{cases} \beta_0^{rw} + \beta_1 m_t + \beta_2 t_{it} + \beta_3 t_{it}^2 + \beta_4^{rw} \mathbb{I}{(s_{it}>\bar{s})} + \beta_{\eta} & \text{if } rw\neq\emptyset \\
			   b & \text{if } rw=\emptyset 
			\end{cases}
			\label{eq:inc}
\end{align}
where $b$ is an exogenous level of unemployment benefits. Apart from the employment status, income thus depends on the specific region in which the individual works in terms of a separate base wage level $\beta_0^{rw}$. Potentially, I could let amenities of the regions such as for example the base wage level follow some stochastic process too. That would allow  agents to move to a given work region also just in \textit{expectation} of earning a higher wage in the coming years. The way it is specified here means people only evaluate the benefits of moving to $rw$ in terms of the current base wage level. On top of this base level comes an age profile to capture the effect from experience. I allow for a non-linear effect and most likely a positive but decreasing return to age. Since I model income\footnote{There is data on hourly wages in the dataset used but these are noisy measures of the actual hourly wage an individual gets. For instance, there would be no records of overtime pay and the overtime hours in general which would tend to bias the estimate of hourly wage. Therefore I use total income which is more reliable since measures come directly from tax authorities.} and not wages and only part of a person's income depends on his place of work, the age components would also capture if a certain age group is more likely to earn income from e.g. investments in shares, transfer payments from the government besides unemployment benefits etc. In addition to age also schooling affects the income level. Here I only consider two levels of schooling, essentially low and high, but more groups could be introduced. Also, the return to schooling depends on the region. This adds flexibility to the model and should be helpful when trying to match the model and the data. If the return to high education is higher in region 1 highly educated people should be more attracted by this work location and therefore face a stronger incentive to work here, all else equal. The model will thus predict that conditional on the other states a person with high level of schooling has a higher probability of choosing to work in region 1. In the data we see that highly educated tend to cluster in specific areas, cf. \autoref{sec:data}, which is something a distinction of returns across regions will help explaining. Including income type $\eta$ means controlling for unobserved individual-specific effects that are constant over time. Some people are just better at negotiating wages, very productive to a degree not captured by e.g. schooling, lucky, have excellent networks, good at investing money in portfolios bringing high returns or are characterized by other constant factors that influence their income in an unobserved way. Individuals know their predicted income before making moving decisions, but the econometrician does not observe the income type. In this model there is no stochasticity in the wages, something that will be changed in future work. However, the model encompasses involuntary unemployment which is a stochastic outcome that affects income so income is still not purely deterministic as long as not conditioning on unemployment type. \citet{Oswald2015} and \citet{KennanWalker2011} allow for income shocks directly, but these are not revealed until after the decisions have been made. 

The remaining components of $u^{d_{it}}(\tilde{x}_{it})$ are specified as follows
\begin{alignat*}{3}
& taste_{it}^{rh} &&= \tau_0^{rh} \label{eq:taste} \numberthis \\
& hcost_{it}^{rh} &&= P^{rh} \label{eq:hcost} \numberthis \\
& swcost_{it}^{rh,rh_{t-1}} &&= \mathbb{I}{(rh_{t-1} \neq rh)}  (\gamma_{j}+ \gamma_1 kids_{it} + \gamma_2 \mathbb{I}{(couple_{it})} \\
& &&+(P^{rh}-P^{rh_{t-1}} ) )  \label{eq:swcost} \numberthis \\
& comcost_{it}^{rh,rw} &&=\psi_0 time(rh,rw). \label{eq:trans} \numberthis
\end{alignat*}
\eqref{eq:taste} is just a residential location-specific constant. This means, I do not explicitly control for such things as crime rates, pollution, quality of schooling, access to green areas or beautiful nature etc. that may characterize the home location, but rather control for these things in a very parsimonious way by putting it all into one parameter. As a consequence, I cannot tell how investments specifically in parks, for instance, would affect location decisions and prices in the area, but since such counterfactuals are not the focus of this paper, it is not too important here. I simply allow home locations to differ in terms of one overall measure. Moving on to \eqref{eq:hcost}, this tells that costs of living in region $rh$ is solely a function of the house prices $P$ in that area. This is in line with e.g. \citet{Oswald2015} and is realistic in the sense that the opportunity cost of living in your house is the money you could get incl. the return on this if you sold or rented it out. Here, I do not explicitly account for the alternative return one could get from investing in e.g. bonds. This is not a big problem, if the focus is on identifying relative prices across regions and if this alternative return affects prices by the same  factor in all locations. \citet{Buchinsky2014} allows housing costs to depend on individual characteristics such as marital status, family size and an unobserved discrete type control, but since I directly observe house prices, I include housing costs explicitly. \citet{Buchinsky2014} also discount housing costs by the factor $(1/1+rp_r)^{t-T}$, where $rp_r$ is the per-period rate of price increases in region $r$, parameters to be estimated. I do not include price increase parameters in the model, first of all because it complicates estimation the more parameters that are to be estimated. I therefore assume that individuals have static expectations about house prices. Modeling house price expectations differently requires these to be part of the state space. Something that would make the model much harder to estimate. On the other hand, contrary to \citet{Buchinsky2014} and the literature in general, I let switching costs in \eqref{eq:swcost} depend on the difference between house prices in the regions from and to which a person moves. $swcost$ thus both allows for an overall psychological cost of moving parametrized by $\gamma_{j}$, where $j\in\{0,1\}$ indexes unobserved moving types, and the trading costs $P^{rh}-P^{rh_{t-1}}$, which are both parameters. I allow for type-specific moving costs to parsimoniously account for the fact that some individuals never move. $j=1$ is the high-cost type and 0 a person with low moving costs. I also control for civic status through the indicator $\mathbb{I}{(couple_{it})}$. The reason is that since I do not model any bargaining between spouses, I may very well observe in my data that people move in order to get closer to their spouse's job. If so, there is a probability that another person in the dataset moves further away from his work region. This may be hard to explain from the other state variables and cause the model to deliver weird parameter estimates. My hope is that by, very parsimoniously, controlling for whether or not the person is in a couple, I can sort out that variation stemming from moving because the spouse is the one who actually optimizes the household problem, not the individual himself\footnote{An alternative would be to model the household problem rather than an individual optimization problem. However, this means I should model the bargaining process of the couples, something that complicates both solving and estimating the model. So far, no one in the literature has succeeded in solving and estimating a structural dynamic discrete choice model of household location decisions in terms of residential location and the work location of both spouses. I therefore save this for future work.}. Finally, \eqref{eq:trans} specifies the commuting costs. These are a function of the commuting time between home and workplace. I therefore allow for commuting costs both if the person does not work in the same region as where he lives and if he does, but it is 0 per definition if the person is unemployed. Therefore, I can both distinguish between the effect of increasing mileage allowance that mainly affects job commuters not living in the same region as the job location, and how e.g. infrastructure investments in the metro affects location decisions. An example of the latter is the expansion of the metro network in Copenhagen by "Cityringen"  that makes it easier to travel in the center of Copenhagen. When solving the optimization problem in \eqref{eq:vphi}, the agent therefore has expectations over $inc$ according to the job arrival density $q$ based on  \eqref{eq:pdis} and \eqref{eq:pjob}, the arrival of kids according to \eqref{eq:kidsdens} and the macro state as stated in \eqref{eq:macrodens}. When estimating the model, I in addition must integrate out the individual-specific income and moving cost types $\eta$ and $j$, since these are only observed by the agent himself. They are both assumed to follow discrete distributions and the probabilities of each value of either control will be estimated along with the other structural parameters at a later stage.

\section{Transition of states}
Most state variables evolve deterministically; the individual ages by one year as long as he has not reached the final age $T$. There is a constant mortality risk of 0 up until that age. $rh_{t-1}$ is per definition the previous home location in the current period. Potential changes in civic status are considered purely white noise and exogenous and schooling is constant. If focusing on adults above e.g. the age of 27 it becomes more likely that people's education does not change in the data either. The macro state on the other hand is assumed to follow a Markov process with transition probability 
\begin{align}
Pr(m_t=n|m_{t-1}=l)\text{ }n,l\in\{0,1\}. \label{eq:macrodens}
\end{align}.
$kids$ can change over time, but since I do not model fertility decisions, I let the arrival of kids be random shocks. Since it is only in rare events that more than 1 childbirth occurs within the same year I consider having an extra child as a 0/1 outcome. The number of kids in the previous period thus affects the number of kids in the current period. Furthermore, because children move out of home at some point, it is possible to go from having a positive number of children in the household to having less. This could be taken into account in the transition matrix by tracking the age of the child. This introduces many more state variables though, something that complicates the solution and estimation. Instead I use the age of the parent, a state variable already included, to predict when children move out of home. I do not fully specify a distribution for the number of kids in the household but just state that it is given by the following
\begin{align}
kids_{it+1} \sim h(kids_{it},t,couple_{it}) \label{eq:kidsdens}
\end{align}

\section{Solving for equilibrium prices}
Until now, prices of housing $P=(P^1,P^2,...,P^{\bar{rh}})$ have been taken as given, where $P^1=1$ is numeraire and $\bar{rh}$ denotes the highest region index. To emphaisze the dependence of the CCPs on $P$, I denote $\Pi(d|x,P)$ as $CCP(d|x)$ for a given value of $P$. To find $P$, I write up the system of $\bar{rh}-1$ equilibrium conditions on the markets:
\begin{align*}
\sum_{rh_{t-1} \neq 2}{\Pi(2|x(rh_{t-1}),P)\cdot w_t^d}&=1-\sum_{rh_{t-1} = 2}{\Pi(2|x(rh_{t-1}),P)\cdot w_t^s} \\
&\vdots \\
\sum_{rh_{t-1} \neq \bar{rh}}{\Pi(\bar{rh}|x(rh_{t-1}),P)\cdot w_t^d}&=1-\sum_{rh_{t-1} = \bar{rh}}{\Pi(\bar{rh}|x(rh_{t-1}),P)\cdot w_t^s} \numberthis \phantomsection \label{eq:eqcond},
\end{align*}
where the left hand sides make up the expected demand for region $rh\in D^{rh}$ and right hand sides the expected supply. $w_t^d$ and $w_t^s$ are the weights of each combination of the state variables and available choice sets in the population grouped by $(rh,rh_{t-1})$. I need these weights to account for the fact that states are not uniformly distributed across regions, and so summing over all states without weighing how many people are in each state in the regions would be wrong; there are multiple numbers for $CCP(rh|rh_{t-1})$ because there is a CCP function for each state and choice set. That is I weigh each CCP used in \eqref{eq:eqcond} with the weight of the state combination representing that specific CCP. The expected demand comes from all those individuals who do not already live in $rh$, but would like to do so, and the expected supply from one minus the sum of the CCPs for choosing $rh$ as the home location for those who already live there. To solve for the $\bar{rh}-1$ unknowns of $P$, I find that $P$ which ensures expected excess demand is 0 in all markets in a given period (calendar year) $y$, i.e. that the $\bar{rh}-1\times 1$ vector
\begin{align*}
ED(P)\equiv \mathbb{E}[demand(P)|x]-\mathbb{E}[supply(P)|x]=0.
\end{align*}
It can be verified that $\Pi$ and thus $ED$ are smooth functions of $P$, so $ED$ is also differentiable in $P$. Thereby, I can use (quasi) Newton-based algorithms to solve for $P$.

\section{Computational challenges}
A challenge of the model presented in this section is the curse of dimensionality caused by the fact that the number of choice sets grows exponentially in the number of locations allowed for. The choice set drawn by the individual in each period in effect becomes a state variable, meaning the model must be solved for extremely many combinations of states. This is not doable within a limited time frame. For instance, it takes about 4 hours \textit{given} a guess on equilibrium prices to solve the model in a version with $T=2$ and $|rw|=|rh|=5$, even after parallelizing as much of the code as possible at the moment. If increasing $T$ a bit, the problem gets very hard to solve and takes several days. It is not surprising that increasing $T$ from 2 to 3 makes a significant difference, since when $T=2$ there is only one period where the expected value term enters the problem, which is what takes time to compute. As a consequence, I cannot currently use the full model for simulations, but save that for future work. Suggestions for how the problems could potentially be solved would be to restrict the number of regions from which an individual can receive job offers in each period. In the above version of the model, everyone can get job offers from the entire set of regions. It is probably unrealistic that people even search hard for jobs in all regions why it also should be unlikely to get offers from everywhere. I definitely do not have as a goal to model the job search process very detailed, but it still might help to restrict the total number of job offers to some number, potentially dependent on individual-specific characteristics. For instance, it is likely that highly educated people use web-based search when looking for a job more than low-skilled people do. The latter might be more exposed to the local job centers and their services of helping finding jobs more locally. It might even be that I would not have to structurally model the process of picking choice sets for each person. \citet{McFadden1978} did indeed show that randomly sampling a subset of the full choice set does not affect the consistency of estimators in the multinomial Logit model with homogeneous tastes for observed characteristics. Whether this applies to this context of unobserved heterogeneity on top of the i.i.d. stochastic error terms is not completely clear at the moment. However, \citet{Keane2012} shows promising results in this regard as they show the bias is limited in the models they consider. Another way of speeding up the computations would be to convert the code to C. It is currently written in Matlab and speed improvements will likely appear when using C instead. 
