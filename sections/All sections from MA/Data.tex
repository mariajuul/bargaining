As previously noted, I I do not aim at estimating the structural model presented in \autoref{sec:model} yet. In the current section I will, however, describe the data that will be used for estimation at a later point. Additionally, I carry -out descriptive and regression analyses to get a sense of the correlation patterns observed. This will help guiding the model set-up such that a serious attempt can be made that those correlations observed can be captured by the model. Overall, I use two different main data sources. One is from Statistics Denmark, from which data on personal characteristics of the entire Danish population since 1992 is delivered. The other is the Technical University of Denmark (DTU)\footnote{With great help from Professor Mogens Fosgerau and researcher Katrine Hjorth.}, where data on travel time and distances between areas of Denmark come from. In the following I will go more into details about the specific data sources used in this project as well as give an overview of the institutional setting related to the paper.

\section{Institutional setting}\label{subsec:instset}
This section will briefly describe the commute tax deductions scheme in Denmark since this is a policy that effectively reduces the cost of commuting. As the model of this paper specifically aims at explaining commuting patterns, it is relevant to understand how these are influenced by governmental initiatives when in future work taking the model to data. Besides, I describe the price ceiling on rent to draw attention to the fact that equilibrium house prices may be affected by this. This is not something the current model can explore per se since there is no choice between owning and renting, but the price ceiling is still relevant to the understanding of the housing market.

%\begin{table}[]
%\centering
%\caption{My caption}
%\label{my-label}
%\begin{tabular}{llllllll}
%           &                  & 2010  & 2011  & 2012  & 2013  & 2014  & 2015  \\
%Number     & Capital Region   & 8,154 & 7,234 & 7,592 & 8,488 & 8,488 & 8,488 \\
%           & Zealand          & 1,698 & 1,517 & 1,560 & 1,795 & 1,795 & 1,795 \\
%           & Southern Denmark & 3,266 & 2,842 & 3,073 & 3,225 & 3,225 & 3,225 \\
%           & Central Jutland  & 3,632 & 3,694 & 3,463 & 3,642 & 3,642 & 3,642 \\
%           & Northern Jutland & 1,220 & 959   & 1,190 & 1,145 & 1,145 & 1,145 \\
%Percentage & Capital Region   & 1.5   & 1.5   & 1.5   & 1.6   & 1.8   & 2.0   \\
%           & Zealand          & 1.1   & 1.1   & 1.0   & 1.1   & 1.2   & 1.3   \\
%           & Southern Denmark & 1.0   & 1.0   & 1.0   & 1.1   & 1.1   & 1.4   \\
%           & Central Jutland  & 1.1   & 1.1   & 1.0   & 1.1   & 1.3   & 1.2   \\
%           & Northern Jutland & 0.9   & 0.9   & 0.9   & 0.8   & 1.1   & 1.0  
%\end{tabular}
%\end{table}


There are basically two types of compensation given to individuals in connection with work-related driving. One is mileage allowance, which is given to employees who drive in their own car during working hours, e.g. to and from customers. In 2016, an employee gets 3.63 DKK/km for the first 20,000 km driven and 1.99 DKK/km thereafter. In order for the amount to be paid out by the employer it must be reported to the tax authorities. These tax deductions are thus not related to the commute between home and work, which is rather denoted commute tax deductions. These are given to people who commute more than 12 km to work. The daily deduction rate varies from year to year and was 1.99 DKK/km for total daily driving between 24 and 120 km and 1.00 DKK/km for km exceeding 120. Since 2004 people living in certain peripheral municipalities (udkantskommuner) are not encompassed by the lower deduction rate after 120 km. This exemption was made in response to an increased focus on stamping out the depopulation of rural areas. Originally, this was a temporary scheme, but in 2006 it was extended until 2013 and in 2009 again extended until 2018. The peripheral municipalities are defined as those with a low wage income compared to the rest of the country and with a low population growth. 16 out of 98 of the municipalities defined according to the 2007 reform of the Danish municipalities fulfill these requirements. Special rules also concern people with low income as their deduction rate can be increased up to 47 pct. in 2016. Since 2014 there has been a tendency for the commute tax deductions to be falling, though they have usually been increased from year to year since the 1990s\footnote{See \url{www.skm.dk/skattetal/beregning/skatteberegning/befordringsfradrag-2016-og-2017}  for details.}.   

Even though the model of this paper does not model the decision to rent vs own one's home, the rental market is still of relevance since the availability of tenancies may affect the market for owner-occupied dwelling. In Denmark, there is a quite strict rent regulation which does seem to be binding at least in the big cities according to the, for the Danes, well-known expectation about having huge difficulties trying to get into the market for rented housing. The Housing Control Act defines how the maximum rent should be determined and the rules distinguish between housing with respect to the construction year of the building, how many leases the building has and how much the lease and the building itself has been renovated. As a rule the rules about cost-related rent apply. In this case the allowed rent is calculated based on necessary and acceptable expenses related to running the building. One can only override cost-related rent regulation if certain conditions are met, for instance if the house was built after 1991. On top of the cost-related rent, the landlord is allowed to add a specified amount to cover the return on his capital and resources to maintain the building and the tenancy. The same holds if the owner has renovated e.g. the kitchen or bathroom. If the tenant finds the rent too high she can lodge a complaint at the rent control board. If the rent control board concludes the rent is indeed too high, the landlord is required to comply with the ruling that the board agreed on. The difference between the legal rent and the one that the tenant has actually paid must be paid back to her retroactively as long as the case about the rent level has been raised at the rent control board in due time, which is 1 year. This also means there is a rather serious punishment of landlords who do not follow the rules in the first place. Some municipalities are exempted from the law. It is up to the municipal council to decide if the rules concern their municipality\footnote{See \url{www.retsinformation.dk/forms/r0710.aspx?id=173134} and  \url{www.boligassistancen.dk/jura-a-k/husleje/} for further information.}. As an alternative between owning and renting a home, Denmark is characterized by a rather high number of cooperative dwellings. These are all owned by a cooperative housing association. One thus cannot buy a cooperative dwelling but only the right to use it by paying a deposit (andelsværdien) to the association which owns all the apartments. It is therefore also the association that makes all decisions regarding the building and the apartments, though the members (who are identical to the users of the apartments) of the association have co-determination. All members are subject to joint and several liability for the debt in the association. This may have serious consequences if being a member of an association with unhealthy finances. If one would like to sell his part of the cooperative dwelling he cannot decide the sales price himself. This is something that is determined by the association and depends on the share of the assets in the association that the apartment represents. In recent years there has been much focus in the media on money paid under the table to the person who wants to sell his membership though it is strictly prohibited. This is due to the big demand for this type of housing which has become very popular especially in the Copenhagen area because it is an easier way to get into the real estate market, especially for first-time buyers since the up-front deposit is usually much lower than the market price of a similar freehold flat. In addition to the deposit one is obligated to pay a monthly amount corresponding to a rent. This is used to cover operating costs and costs to pay down potential loans in the association. On top of that comes council council flats (almene boliger) which are regulated by the law about council housing. These are houses built with financial support from the municipalities and owned by a housing association. In return, the municipality typically has every fourth of the homes at its disposal, often used to help people and families who are not able to find affordable housing by themselves. The rent in such houses or apartments is also relatively low and cost-related. The association that owns the homes are thus not expected to earn any profit from the rental service. This type of housing is for everyone but built with the purpose of helping population groups with special needs. These include young people enrolled in education, single parents with children, the elderly and disabled people. To get such a dwelling one must be on the waiting list which can be very long. For many years there was a tradition for parents to put their newborn's name on the waiting list, but the rules have been changed such that one must be at least 15 years old to get on the list. As mentioned above, the model of the current paper does not distinguish between owning, renting or living in a cooperative dwelling. However, these distinctions are important to keep in mind and may be relevant at a later stage where the model is estimated and hence supposed to be able to fit the data sufficiently well. Especially because prices differ across the different housing types.

%There are 3 types of cooperative dwellings: 1) traditional private cooperative dwellings where the association has bought an existing building. 2) non-subsidized private cooperative dwellings where the association can build new properties without any governmental support and 3) subsidized private cooperative dwellings where the apartments are newly built and erected with support from the government after 1980 but before 2004. 
\section{Data sources}
The data from the administrative registers at Statistics Denmark contain information on the entire Danish population from 1992-2011. The population register BEF holds data on a masked version of the social security numbers of all individuals with residence in Denmark according to the National Registers of Persons (Folkeregisteret). This is a key variable since it enables me to link much information from other registers together. The model is one for households, where the household head decides on behalf of the family\footnote{I therefore use ``household'' and ``individual'' interchangeable since in essence it is only one person deciding.}. In the BEF register, there is a family identifier where a family can be either a single person or a couple with or without children. Children living at home belong to their parents' family as long as they live at the same address as one of the parents, are below 25 years old, have not been married or lived in a civil partnership, do not have children themselves and do not make up one part of a cohabiting couple. The family id is stable over time as long as the person either stays single or the two people in the couple stay as a couple. If a couple splits up, both of them will get a new family id as long as they do not still live together. The same holds if one of the partners die. In addition to family ids, BEF thus also tells whether the person lives in a couple or not. A couple can be either a married couple, a civil partnership or cohabiting people with or without children\footnote{See  \url{www.dst.dk/da/TilSalg/Forskningsservice/Dokumentation/hoejkvalitetsvariable/familier/familie-id} for more information on the identification of cohabiting couples who do not have children together.}. From BEF I also get the address of each person in the year. The addresses consist of identifiers for street name, house number, floor and door side and are unique within a municipality, so combined with information on municipality I get the exact address of the individual. In the regressions in \autoref{sec:descriptives} I use the new municipalities from the 2007 reform. The old municipalities can be converted to new municipalities\footnote{With the reform 271 municipalities were merged into 98, though there were a few old municipalities that were split into 2 or 3 new municipalities. In the conversion I let each old municipality belong to only one new municipality. More details on the reform is presented in \autoref{sec:descriptives}}. The timing of moves can be identified from the date of official change in address. The law requires people to let the municipality know about their change of address no later than 5 days after the move. There is a fine for not complying with these rules and since not registering one's new address means mail is not delivered at the new home, very few people probably do not change their official address very fast. People who move within a year appear with several homes in that year. I need to structure the dataset such that there is only one home per year since the time frame in my model is yearly. Different solutions can be suggested. Statistics Denmark use the residence by January 1st as the home, but I decided to use that address where the individual lived for the main part of the year. In addition to residence information, other personal characteristics from BEF that I use include age, gender, parish in which one was born (useful for individuals who were born before 1980 where BEF was started) and lastly mother's and father's social security numbers. From the income registers INDH and INDK I get information on wage income, total income, commute tax deductions, wealth and debt. 

The population register can be merged on to the Integrated Database for Labour Market Reasearch (IDA). This is a panel of all employments regarding persons living in Denmark by the end of the year since 1980\footnote{The original database was stopped in 2004 where the 12 datasets were instead combined into 4 datasets which were followed back to 1980. The four datasets contain information on individuals, employments, workplaces and firms, respectively. These are merged via an identifier for the workplace.}. The database allows me to link individuals and firms and get data on the start year of the employment and number of days employed by the employer who is also occupied with an employer id. These are based on the start and end dates of the employment that come from the Central Tax Information Sheeet Register (Centrale Oplysningsregister) until 2008 and from eIncome which is also located at the tax authorities. Individuals can have several jobs during a year. The register is made up in November of the year and uses the register-based labor force statistics (Registerbaseret Arbejdsstyrkestatistik) into either employed wage-earners, employer (A), self-employed (S) or co-working spouse (M). These four groups are mutually exclusive and the difference between A and S is that being a type A means having employees contrary to S. The category of employed wage-earners can be further divided into main occupation (H), sideline occupation, another November occupation, and most important non-November job. The two latter, however, are only defined from 2004 onwards. The type variable is important when determining which job is the main job to which the individual spends most of the time commuting to. I restrict attention to H and A jobs since the simultaneous home and job moves may be very different for people with S and M jobs since the job in essence can just move with the person. It may be, however, that one's H or S job in November is not the job that the individual has had for the major part of the year. In that case the most important non-November job should be considered the job of the year. Since this type is not defined until 2004, I looked at data from 2004-2013 to check how restrictive it would be to define the job in the year as the H or S job no matter what other job categories might be present. I found that for 90\% of the population the most important non-November job had been the main employment for less than half a year and vice versa for H and S type jobs. I therefore decided to use the November employment to define the job of the year. In general, I aim at defining home and job location such that the probability that I model the commute that took place during most of the year is high. In this regard there is a trade-off since I also want consistency in the data which is why I do not exploit the non-November employments from the point in time where it was defined. In order to be able to model the commute I need information about the location of workplaces. Fortunately, the database allows me to not only link employees and employers but also employees and workplaces (and workplaces and employers). The workplace has an address code attached to it from which I can get the municipality. In those cases where an employment cannot be assigned to a registered workplace it will be assigned a so-called fictitious workplace and the address will be the residence of the individual. This is often the case for people who conduct their work from or near their home or at several different workplaces. The latter concerns, in particular, workplaces for cleaners, insurance and for people working in the social- and healthcare system as for instance a community nurse\footnote{See \url{www.dst.dk/da/TilSalg/Forskningsservice/Dokumentation/hoejkvalitetsvariable/ida-arbejdssteder/lbnr} for a more elaborate explanation.}. Of course this gives rise to problems when calculating the travel time or commute distance for these workers and is something one must have in mind. This can be regarded a measurement error in the workplace variable. The workplace variable is used to get travel times between residential and work location. The travel time estimates will thus tend to be downward biased for people with fictitious workplaces. 

The data on travel times come from Landstrafikmodellen (LTM) developed by researchers at DTU. In LTM, Denmark as a country has been divided into 907 zones\footnote{There are 4 different zone levels in LTM. This corresponds to level 2, which is rather detailed, and still ensures that the data complies with Statistics Denmark's rules about discretion. See \url{www.landstrafikmodellen.dk} for more information.}. The number of trips by use of different transport modes between pairs of zones are estimated in the model. The definition of zones are based on the parish borders which can be linked to municipalities and provinces (landsdele). This is useful since the model I currently use cannot be solved for too many locations, something I will elaborate in \autoref{sec:model}. However, this causes a challenge when transforming zone level travel times into municipality level travel times. First of all, travel times between two zones are averages of travel times over all trips between the two zones in a given year. There are different definitions of travel time, namely both by public transportation, car and walk or bike. Travel time by car is given by the sum of free time (minutes in car with free flow, i.e. where the speed equals the allowed speed), congestion time (minutes with congestion, i.e. where the speed is less than the allowed speed), ferry time (minutes sailing by ferry), ferry wait time and pre-departure arrival time that take wait time into account. Travel time by public transport is given by summing waiting time, walk time in connection with shifts between different buses, trains and other public transport modes, walk time to and from first and last stop, respectively, and lastly travel time by other vehicles. In addition, travel time by walk and bike is calculated by LTM for pairs of zones where walk and biking trips actually exist according to the model. It is important to note that the LTM has been run for 2002 and 2010 only. Walk and bike time is invariant, but travel time by car and public transportation may change over the years. This is due to for instance new stations/stops being established or existing ones closed, just like construction of a new high way influences travel time. It is very complicated to run the model for a new year and I will therefore make do with the two current versions of the model. To compute travel times between municipalities instead of between zones from LTM I follow the recommendations from researchers at DTU and calculate a weighted average of travel times using the zone pairs within the municipalities in question and weighing travel time in a given zone pair in the municipality by the number of trips made according to LTM. This ensures that when someone is observed to live in municipality A and work in municipality B, I assign that travel time that the average trip takes when accounting for the fact that it is most likely (unconditionally) that this person starts and ends her trip in the zones characterized by most trips. To get the travel time for a given person and year in the dataset, I calculate both travel time by public transport, car, walk and bike for 2002 and 2010. I want one measure of travel time only for each person in the year and use the minimum of all 4 travel times as the representative travel time. This is done both for the 2002 and 2010 versions. The line of thought behind this rule is that I do not observe how people commute. I could, in principle, observe if a person owns a car and thus decide if it is likely that she commutes using the car. However, since cars can be shared within a household and most households own only 1 car, if any, I would have to guess which of the household members used the car. Also, since there can be a huge difference in travel time when using car instead of public transport it would not make much sense to use the mean of the two travel times when calculating travel time by transport. Of course, by taking the minimum of the two travel times, I do underestimate travel time for some people. However, if there is a very huge difference in travel time, from pure intuition it makes sense to assume the fastest mode is used. On the other hand, if the two travel times are not too different, the mistake is not too serious. Another argument for using the minimum of the travel times is that it represents the fastest possible way to get from home to job. I therefore implicitly assume that this is an amenity that people attach to the locations and base their decisions on that rather than also considering whether to buy a car, go by train or bus or choose to walk or bike.

LTM also provides a measure of the distance. In addition, a work distance variable is available and calculated based on actual work and home addresses, but only available for 2000-2008. As above the actual work address is however unobserved for those who do not have a registered address because they do not have a formal workplace but rather drive around the country to sell e.g. insurance products. This work distance variable should be highly correlated with travel time. This is indeed the case with a correlation coefficient of 0.76. There is a choice between using work distance vs travel time to measure the burden associated with the workplace. The arguments for using travel time is that two jobs located in the same distance from some home location may give rise to very different travel times dependent on congestion and public transport availability. On the other hand, travel times are only available for 2002 and 2010 and are averages over zone pairs within the municipalities for home and work. %on commute tax deductions from the tax authorities. These tax deductions are observed in the registers and comply to every employed person who lives more than 12 km away from his workplace. Until a commute distance of 100 km, there is a base deduction rate that varies from year to year. Hereafter, the tax deduction rates is half of the base rate\footnote{\textcolor{red}{}See [cite appendix] and \citet{Munk-Nielsen2015} for details.}. Using these rules, a measure of the work distance can be obtained. Note, however, that it is not the exact work distance for each individual. For those who commute less than 24 km per day (to and from work), the tax deductions are 0 and work distance set to 0 as well. Moreover, I observe total deductions which are a computed as the deductions per day times number of days worked. The latter is unobserved why it has been set to 225 days in all years for all persons. This corresponds closely to the the official number of work days in all years. however, the correlation increases markedly to \textcolor{red}{0.89}. This is likely an artifact of the above-mentioned problem with people who do not have a registered workplace as these people's workplace is set to the home address. If they actually do not work in same municipality as they live but rather drive around the country to sell insurance products for instance, their commute tax deductions and thus the work distance variable will reflect this. 

At this point I do not have access to much data about the houses. I do observe whether a given person owns a house, though, but not which house. When accounting for home ownership I therefore assume that if someone owns a house, the house that he or she lives in, is owned by that person. It might be that you are a home owner but have rented out your home and rented another home for yourself. The data cannot tell anything about this at the moment, but I suspect it is a rare situation. At some point in the near future I will have more detailed information about ownership and which exact house someone lives in. Also data on characteristics of these houses are available including transaction prices of all dwellings in Denmark as well as public valuation estimates of all houses. This will turn out to be very useful when estimating the model and finding equilibrium prices at the housing market.


%\begin{itemize}
%\item The data I use come from the Danish registers. These contain information on every single individual in Denmark and a huge amount of characteristics about these. 
%\item BEF: residential info (regional, municipal, parish or zip code level) and the exact moving in dates.
%\item BOL: $m^2$ of house, \# rooms, whether house or apartment, rent(?) year of construction.
%\item IDA: job industry, occupation, employment period, employment location.
%\item INDH: income, commute tax deductions, disposable income, rent subsidy, wealth, debt.
%\item EJSA, EJSD: info on sold property, incl. the sales price. 
%\item EJER: info on owners of houses.
%\item EJVK: assessment of property for tax authorities.
%\item (Data on ranking (GPA, student well-being) for all Danish schools from Ministry of Children, Education and Gender Equality.)
%\item (Data on amenities of regions like public level of service, crime rates, etc.)
%\item Highly educated cluster in big cities. Probably because jobs are located here - look at jobnet data? I do not model that.
%\item Institutional background: transport allowance, price ceiling on rent, moving jobs out of capital region, valuation of properties.
%\item Moving costs = 4 pct. of house value according to Simon's PhD report.
%\end{itemize}
