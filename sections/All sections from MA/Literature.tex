This paper builds on and contributes to three different literatures: 1) residential/work location models, including mostly static equilibrium sorting models and dynamic models for choices of either residence or work location, 2) numerical and theoretical models of equilibrium in markets for durable assets, and 3) structural estimation of dynamic choice models, including dynamic discrete choice models applied to choice of residence. I provide reviews of each of these literatures below, and show that this paper contributes to the literature by being the first dynamic equilibrium model of simultaneous choice of both residence and work location as well as occupational choice.

\section{The start: the monocentric city model}
The literature on household location decisions is based on theory and methodology developed in industrial organization and labor economics in particular. The literature deals with sorting, i.e. with the mechanism that market forces make people with similar preferences and personal characteristics self-select and cluster in same locations. The urban economics literature as a separate research field dates back to a number of seminal papers. It was the theoretical work by \citet{Hicks1932} and \citet{Sjaastad1962} that really made economists interested in understanding the driving forces and implications of how individuals locate, but other models led to and were part of the reason for this rising interest. \citet{Tiebout1956}, who was the first to argue that when people sorted (``voted with their feet'') in terms of residential location they implicitly revealed their demands for the local public goods that were exclusively available for the location in question. He focused on the effect that fiscal competition had on income sorting between jurisdictions. At the same time \citet{Alonso1964} developed the monocentric city model, which was enriched by \citet{Mills1967} and \citet{Muth1969}. In contrary to the Tibeout model, this is a model for income sorting across geographical space and has become the main base for many analyses of locations within a city. The main idea of this model is to consider job locations to be exogenous at the center of the city and reducing the travel problem to that of commuting to one's job, thus ignoring any other travel time and distance that the individual might use to decide on his optimal location (e.g. travel to shops, family or daycare). In a strict sense, the model thereby takes as given that people like to live in big cities, but does not explain why, except the fact that commute is shorter since all jobs are to be found at the Central Business District (CBD). It focuses on the trade-off between living close to the CBD at the cost of more expensive housing. Overall, the consumer maximization problem is a standard one, except that besides choosing an optimal amount of housing and a composite good, consumers also choose a residential location (a distance from the CBD). The housing prices at each location respond such that they exactly offset the marginal decrease in utility that stems from living further away from the CBD. Thereby, an exact relationship between distance to the CBD and housing prices exist. Besides modelling the consumer behaviour, also a construction sector which builds houses by use of land and capital is part of the set-up. This means that also land prices are endogenously determined in equilibrium and the model predicts that as land prices increase closer to the CBD, construction firms tend to build with a higher capital/land ratio, i.e. to build tall apartment blocks rather than big mansions with lots of garden space. 

Extensions of the monocentric city model framework include work by, among others, \citet{McMillen2006a} and \citet{AhlfeldtMcMillen2015} who look more into intensity of development and the \citet{OgawaFujita1980} model that endogenise the location of firms as well to explicitly model agglomeration economics - another branch of the urban economics literature that will not be touched upon in this paper. However, it is still relevant to the model developed in the current thesis since this is the first model that allows consumers to both choose between residential and job location, where the latter is characterized by a trade-off between wages and commuting costs. 

The general scope of the monocentric city model is to predict how geographical space is divided between residential and land use and thus to predict the size of urban areas. The urban area increases until the marginal value of devoting more land to this equals the marginal value of increasing agricultural land use. The most prominent recent paper within this branch of the literature, whose sub-branch is more economics of density rather than actual sorting,  is \citet{Ahlfeldtetal2015}. They set up a general equilibrium model of internal city structure where people select a residence-job combination and where wages and prices of land adjust in response to the moving patterns. Their focus is on estimating the extent of agglomeration on productivity, not on the location choice per se. To disentangle the effect of locational fundamentals on productivity they use data from the division and reunification of East and West Berlin. Thereby they get exogenous variation in the latter. However, they only have data from the years 1936, 1986 and 2006. So even though they do study how equilibrium land prices change in response to altered moving patterns, they can only estimate the long-run effects in a static modeling framework. 

\section{Other reduced-form models}
The discrete choice framework that has been increasingly popular in recent decades, and to which this paper belongs, was initiated by \citet{McFadden1974}. He provided a methodology for analyzing choice behaviour when the optimization is with respect to choices on a discrete set rather than a continuous one as in the monocentric city model approach. He was also the first to really contribute to the discrete sorting literature by applying this method to choice of residence in \citet{McFadden1978}, where people choose a specific location rather than choosing to live on a line with a certain distance from the CBD. 

Another distinction within the location choice literature is that it originally centered around two types of models: human capital on the one hand and hedonic models on the other. Human capital models are, among others, motivated by \citet{Topel1986}. Hedonic models were introduced around the same time as McFadden came up with the discrete choice estimation methods, namely by \citet{Rosen1974} and further extended by \citet{Roback1982}. \citet{Rosen1974} set forth a method for estimating marginal willingness to pay (MWTP) for goods for which there were no formal markets such as air pollution, crime rates and scenic views. Introducing hedonic price functions, he explained how researches could use data on observed location choices by households and housing prices for the different locations to compute implicit price indices of these non-traded amenities. Whereas the human capital models argue that migration occurs due to disequilibrium in the labor market, where people move to a new location to earn a higher wage, the hedonic approach asserts that individuals might move even if housing and labor markets are in equilibrium. The reason being that they have changed their demand for these non-traded amenities. The human capital literature sees earnings differentials as only temporary circumstances since workers will mitigate these when they relocate in order to get the highest possible return on their investments in human capital. However, hedonic models explain how wage and housing price differentials may not be completely eliminated as they compensate individuals for the location-specific amenities and disamenities. With the emerge of the sorting literature, these two approaches were combined into a unified framework that took both mechanisms into account as \citet{Clark1991} in a reduced form model showed to be important. \\

As time has passed, the literature on migration or location choice models per se has become vast and has been surveyed by among others \citet{Greenwood1997} and \citet{Lucas1997}, the latter focusing on cross-country comparisons. However, for a very long time these papers have indeed been of a reduced form type, the monocentric city model included, and thus have not been able to explain the underlying behaviour of the decision makers, but have rather described migration patterns as they were observed in the data. \citet{Bartel1979} is one such example. She analyses the determinants of migration in terms of job prospects and the consequences of migrating on your labor market success. \citet{Graves1979} use a Probit model to relate demand for non-traded goods to optimal relocation. 

\section{Structural static models}
There are examples of papers that do structurally model the location decisions, but they have until recently mainly used static models. Among these are \citet{Borjas2000}, who looked into how immigrants affect the equilibrium in the local labor markets across several geographic areas. The paper points out that the possibility to move to another location for work among natives is not sufficient to cancel all wage differentials across locations because these people have high moving costs that make them inclined to not move for the best wage offer. On the contrary, immigrants from other countries do not to the same extent incur moving costs on top of those associated with leaving their home country and are therefore more prone to settle and work in the area characterized by the best wage offer. The model is static since people do not make several moving decisions, but rather stay in the location where they decided to settle when they first arrived to the country. \citet{Bayeretal2009} is another example. They conclude that MWTP estimates are biased if households cannot move freely between locations, for instance due to costs associated with moving to another place. Consequently, they combined the frameworks originally presented in \citet{McFadden1978} and \citet{Rosen1974} by first structurally modeling the household location decisions taking moving costs into account instead of relying on the first-order conditions from the traditional hedonic model.  

\section{Structural dynamic models}
The reasons that the literature has so far mainly considered the moving decisions to be static are the computational difficulties of allowing for a dynamic consideration as well as lack of data that are rich enough to actually study the dynamic behaviour. Dynamics are crucial, however, as outlined in \autoref{sec:intro}. Using data from the National Longitudinal Survey of Youth 1979 (NLSY79) \citet{KennanWalker2011} were the first to add dynamics to a structural model where individuals optimize over a number of residential locations and where individual choose their optimal residence in each period. There were a few predecessors in the dynamic location choice literature such as \citet{Holt1997} and \citet{Tunali2000}. However, they both did not distinguish between alternative locations, but modeled only the move-stay decision. \citet{Dahl2002} did do so by allowing individuals to choose between all US states, but individuals only made one moving decision for their entire life. \citet{Gallin2004} looked into how changes in expected future wages affected net migration in an area, but he used aggregates and thus did not model how the individual responded to this. Lastly, \citet{Gould2007} studied how workers choose between rural vs urban areas as well as their career in a dynamic structural model to be able to decompose the urban wage premium into factors stemming from productivity differences and non-random selection of workers into urban area in terms of their ability and endogenous past career choices. However, the model only distinguishes between rural and urban locations, not on a more detailed level. 

So even though work had been done in the dynamic location model literature, \citet{KennanWalker2011} were the first to broaden the application to a more detailed setting, where they use expected income changes as the main driver of migration decisions. They allow for many different locations (US states) but restrict people to live and work in the same location. To deal with the huge state space they restrict the choice set to only include a restricted number of locations, where the individual is likely to have good information about the wage potential. In effect, they regard wages to be experience goods, i.e. you do not know the wage potential of a region until you move there. This also means individuals move only in expectation of job offers. They argue that this approximation is not too unrealistic since individuals tend to have better information on the income prospects in locations they have or have had some connection to. They conclude better income prospects are important drivers of migration decisions. \citet{Bishop2012} has another focus, namely to set up an equilibrium model to estimate willingness to pay for air quality while controlling for moving costs and forward-looking behavior. She takes the same stand as \citet{KennanWalker2011} with respect to information about wages when making one's decision, but lets individuals be forward-looking with respect to local amenities such that they can move in expectation of these being altered. Her model can comprise a huge choice set and she employs it to 59 locations mainly defined by metropolitan statistical areas (MSAs) and also estimate the model on data from NLSY79. \citet{Bayer2016} take up her approach, but goes one step further and estimate the willingness to pay for several unobserved amenities of a neighborhood by use of a semi-parametric estimation technique. Additionally, the paper allows for household wealth to evolve endogenously with housing prices and therefore also lets households have expectations about future housing prices when making their moving decisions. Location choice hence becomes dynamic both due to moving costs and wealth accumulation. Consequently, they also allow households to move to get to a region where housing prices are currently low in expectation of seeing increasing prices thereby implying an increase in wealth also. 

Related to this paper is \citet{Hviid2015} which has two chapters concerning dynamic residential choice models. The first of these also looks into the wealth accumulation aspect of the housing market by exploring how home ownership and residential location decisions affect the wealth accumulation. He does this by building a model based on \citet{Bayer2016} but adds borrowing constraints to the household optimization problem. More specifically, he investigates how endogenous income from housing capital and moving costs differ across the wealth distribution and whether any potential differences in these two measures impact the overall wealth inequality. He finds considerable differences in capital gains from home ownership across the wealth distribution and that these amplify existing wealth inequality. The other chapter of his thesis is also very related to \citet{Bayer2016}, as he estimates MWTP for neighborhood amenities based on rich Danish data. Again in a dynamic residential location choice model. He focuses on the case of house size and air quality but in particular studies how credit constraints affect the MWTP too. On top of this the paper demonstrates the importance of accounting for expectations to household size when modeling household location behaviour. 

Until now the presented papers have considered the household head or the individual as the decision maker. \citet{Gemici2011} breaks with this tradition by modeling the bargaining over optimal residential locations between spouses in a household in a dynamic setting. The model lets couples make decisions on consumption, employment status, location and divorce each period. She therefore considers a joint discrete-continuous choice problem and uses the concept of Nash bargaining with a threat point defined as the value associated with being divorced. The results show that family ties do affect mobility in a downward direction and the same for wage growth, especially for men. \citet{Winkler2011} is closely related to \citet{Gemici2011} since he uses the same general approach as her, though without allowing for intra-household bargaining. On the other hand, he adds the choice between owning and renting one's home and is the first to simultaneously model housing choice, residential location and labor market outcome in a dynamic structural model. 

Another recent contribution to the literature is \citet{Oswald2015}, which is also related to \citet{Winkler2011}. In a life cycle model of consumption, choice between owning and renting and residential location he allows for both aggregate and region-specific shocks to affect these decisions. Whereas there has been a tradition for considering people as either home owners or renters, he integrates this decision into the model, since it has a huge effect on the likelihood of migrating. The reason is that owners' wealth declines when house prices do while renters may benefit from lower rents. He argues this is important to understand as the option of moving is a way to self-insure against local shocks to the housing and labor market and focuses on estimating the value of this self-insurance mechanism. As in \citet{Bayer2016}, he allows characteristics of regions (in his case housing prices) to follow different trends and lets individuals have expectations about these. This means people do not only move due to randomly receiving a high enough preference shock as in e.g. \citet{KennanWalker2011}, but may also decide to move because regional characteristics seem to develop in a favorable way. Together with the discrete-continuous choice framework this makes the model computationally hard to solve and estimate why he as a consequence must define regions in a relatively broad sense by using US census divisions of which there are 9. Moreover, the current version of his model does not explicitly account for other region-specific amenities besides housing prices and wages. 

\citet{Ransom2016} takes another standpoint on the migration decisions. Using a dynamic residential location choice model, he rather examines why it seems that the trends in migration for unemployed vs employed workers differ over the business cycle and how the migration rates would be affected by various policies. He is the first to model the joint decision of residential location and labor supply in a dynamic model, where shocks affect the optimal behaviour differently for employed, unemployed and non-participating workers. Among other things he documents that employed people experience a high job queuing penalty when moving and were for that reason more likely to stay in their current job location. This causes job offer and destructions rate to diverge and hence employed workers to react differently, with respect to migration, to labor market shocks than do idle workers. 

\citet{Schmutz2015} also build a dynamic model of job search and add equilibrium constraints on size of the labor market (population size), level of unemployment and the wage distributions. They study how spatial matching frictions and moving costs hamper the job search and ultimately how the coexistence of labor markets with very high and low unemployment rates observed in reality despite huge advances in transportation and communication technologies persist and can be regarded as an equilibrium. Additionally, they aim at disentangling different components of the moving costs to better explain why these are present. They find that mobility costs increase fast with distance between locations and that dissimilarity between sectors, and the implied adaption costs, is a much larger part of moving costs for employed workers than is the case for those without a job. Their paper also relates to the economics of density literature touched upon earlier since they use the model to simulate the optimal size of the urban area. They find that larger cities allow for a more dynamic labor market, but on the other hand cause large spatial frictions due to long distances between cities. 

\citet{Mangum2015} is another example of a paper that models the dynamic decision on where to work (live) with the aim of understanding the mechanisms that cause labor market dynamics to be hampered. It is a spatial equilibrium search model of infinite horizon where workers choose location in each period. Hereafter, they are matched to jobs and produce some output that is afterwards consumed. There is competition about jobs and housing, and while wages are exogenous housing prices adjust to clear the real estate market and so do employment rates in order to clear the labor market. With respect to the housing market, residents are assumed to be tenants who rent the home from a landlord who supplies housing according to some function that is constant over time but may differ across locations. Endogenous wealth from home ownership is therefore disregarded. Also the firm side is modeled, where there are mechanisms ensuring that firms (and hence job postings) enter until a zero-profit condition is met. On the other hand, however, workers are ex ante identical and the only heterogeneity is the one that stems from their different initial location choices that, due to moving costs, cause utility from a given location to differ across workers. To reach a rational expectations equilibrium in the model, workers form expectations about the strategies of all other workers in the economy, because the population size in each location influences the chances of getting a job there and the rental rate. In equilibrium these expectations are fulfilled and found by solving a fixed point equation in the population distribution over locations. The problem is difficult to solve because every state variable in every location enters as a state variable in the optimization problem. For that reason Mangum uses an approximated version of the dynamic equilibrium where states of other locations are reduced into one state containing a summarized measure of the states in other markets. This improves the feasibility of the model such that he he can use the largest 30 US MSAs and does not have to rely on even larger regions in order to solve the model. The paper concludes that labor mobility would be less impeded if migration was not motivated as much by idiosyncratics or if local productivity shocks were not capitalized into housing prices. Nevertheless, the results suggest that there is not much scope for improving aggregate productivity by inducing people to reallocate more optimally by use of certain policies. This paper is of importance especially because it allows housing prices to be endogenous. However, it is still - like the main part of the literature - restricting individuals to reside and work in the same region. 

\citet{Buchinsky2014} is the first to model the joint residential and job location decision in a structural, dynamic model. Also the decision about in which occupation to work (unemployed, blue collar and white collar) is modeled. The model mainly builds on \citet{KennanWalker2011}, but a very important extension is that individuals are not imposed to work and live in the same region. Hence, individuals in the model choose home and job location as well as labor market status and sector each period. By relaxing the assumption of zero commute they can model commute costs. In their paper, they do that in a rather parsimonious way by estimating a parameter for a dummy of commuting. Very importantly though, they allow people to know about the job offerings before they decide on their locations. This complicates the empirical implementation of the model since job offers are not observed, but also makes it more realistic. Wages are therefore known before moving. As in \citet{KennanWalker2011} they specify a Mincer-type wage function where, besides individual observables, individual-, region- and occupation-specific shocks affect wages and are serially correlated . However, this is the only shock in the model and all idiosyncratics are thus specifically tied to the income process. In other words, the shock is more of an income shock than a taste shock which is the standard interpretation of shocks in dynamic programming models. Unless wages capitalize changes in amenities or characteristics not directly related to the job, e.g. lower crime rates, more kids in the household etc., the model cannot tell how individuals incorporate uncertainty about these features into their decision making. Even though the model is very rich in terms of its choice set, it is a partial equilibrium model where both wages and prices are taken as exogenous. They argue this is not too important for their setting, where they use survey data on immigrants from the former Soviet Union into Israel. This data does provide the authors with quite detailed information about the households, and even though bargaining is not taken into account they do try to control for the effect that cohabitation may have on location decisions by including spouse variables in the utility specifications.To estimate the model the paper applies a method of simulated maximum likelihood, where several choice histories for each individual are simulated and subsequently matched to the observed choice history in the data to estimate the parameters of interest. One limitation of the paper is that it restricts attention to the case of hihgly educated immigrants from the Soviet Union migrating to Israel, which is likely to be a very selected group. Their results also mainly regard immigrants contrary to people who migrate within a country.  

The contribution of the present paper is thus to set up a model of home and job location decisions and at the same time allow housing prices to be endogenous. This is not a trivial extension however. The model will be used to make counterfactual simulations of policy experiments regarding commuting. Ultimately, I will look into how commuting will be affected to and from urban vs rural areas if transport allowances are increased or how faster public transport between rural and urban areas affects location decision compared to the effect on these when when moving jobs to rural areas which has recently been done by the Danish government. Nevertheless, in this version of the model, I solely simulate the effects of a change in travel time.

%\begin{itemize}
%\item PhD thesis Simon Juul Hviid
%\begin{itemize}
%\item Endogenous wealth accumulation when considering people home owners.
%\item How different initial conditions affect capital gains and moving costs across wealth distribution.
%\item IDA seperates earned income from other income. 
%\end{itemize}
%\item Modibé's paper on job search
%\item The paper on city structure
%\item Sorting literature - voting with your feet Tiebout 1956, McFadden 1974 (method) + 1978.
%\item Review: Kuminoff, Smith + Timmins 2013.
%\end{itemize}

