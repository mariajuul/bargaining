This section will exemplify what the model of this paper can be used for. In particular, I focus on how changes in travel time affect location decisions and equilibrium prices on the housing market. Due to the computational challenges mentioned in the previous section, this section will simulate and interpret the CCPs and equilibrium prices in a very simplified model for a number of scenarios. One should therefore have in mind that the size of the effects by no means are supposed to be interpreted as the effect one would experience if conducting the same experiment in the real world. The signs should be informative though. 

\section{Simplifying assumptions}
To make the model feasible for simulations I therefore simplify it in a number of ways. First of all, I assume there is an employment probability of 1 in all regions. Consequently, there is no risk of being fired so $P^{dis}_{\eta}=0$ and $P^{job,rw}_{it}=1, \forall (rw,i,t)$. People can still choose to be unemployed. This means I take a more neoclassical view where it is always possible to get a job if you want one. Also, I do not model the marginal utility of income as $\kappa^{mu}(inc,m)$ as specified in \eqref{eq:u}. Instead I use 
\begin{alignat*}{3}
&u^{d_{it}}(x_{it})=&&\sqrt{{inc_{it}^{rw}}(x_{it})}+taste_{it}^{rh}-hcost_{it}^{rh}-swcost_{it}^{rh,rh_{t-1}}(x_{it}) \\
& &&-comcost_{it}^{rh,rw}. \label{eq:u_alt} \numberthis  
\end{alignat*}
and thereby impose positive, but decreasing marginal utility of income. Since unemployment no longer affects the probability of getting job offers, unemployment does no longer have severe future consequences. To avoid a situation where unemployment is actually not very harmful except for in the current period I rather add a dummy $\mathbb{I}{(rw_{t-1}=\emptyset)}$ to the $inc$ function and thereby more indirectly control for the effect that unemployment may have on one's future income prospects. 

I solve the model for given parameter values and for $\bar{rh}=\bar{rw}=3$ regions with $T=2$. As mentioned before, even with fewer regions, increasing $T$ makes a huge difference for the execution time. In general, the parameters are chosen such that the CCPs associated with these do not put too low probability on too many cells. For instance, I experienced the solution was quite sensitive to the value of unemployment benefits $b$ compared to the costs of commuting. The solution quickly changed from putting (close to) only positive probability on $(rh,rw)$ with no or very little commute, i.e. work choices involving working only in the home region or being unemployed, but generally the parameter values are quite arbitrary. To have some application in mind, I think of the 3 regions as the province Copenhagen city, Copenhagen surroundings and the rest of the provinces on Zealand (Northern Zealand, Western and Southern Zealand and Eastern Zealand). On top of that comes the work region 4, which represents unemployment. I do not really use much data information on these regions except that I compute average travel times within and between these regions and use these measures in $comcost$. The same with the transition matrix for kids. Since the latter depends on age and maximum age in this model is 2, I decided to use data for 26-35 year-olds, where I defined $t=1$ as 26-30 and $t=2$ as 31-35 years. Then I computed the transition matrix non-parametrically for each age group. See \autoref{tab:apkids}. I disregard the effect of macro state on income for now and therefore do not make an attempt to estimate a transition matrix for that variable. The same holds for many of the other parameters. This is because I use this section to build up intuition for effects on location decisions and equilibrium prices from changes in certain parameters. To be able to isolate the effects from different changes in parameters, I focus on only a few changes at a time. The parameters that I use for each of the 4 scenarios are presented in \autoref{tab:simcoef}. For scenario 1 I do not use the travel time data, but just let all region pairs have a travel time of 1. 

\renewcommand{\arraystretch}{1}
\begin{table}\centering
\caption{Coefficients used in simulations}
\label{tab:simcoef}
\begin{tabular}{@{} l r r r r  @{}}
\toprule
        &\multicolumn{1}{c}{Case 1}&\multicolumn{1}{c}{Case 2}&\multicolumn{1}{c}{Case 3}&\multicolumn{1}{c}{Case 4}\\
\midrule
$\beta_0^1$ &1  &1 &  3 & 3  \\
$\beta_0^2$ &1& 1&    2 &  2 \\
$\beta_0^3$ &1 &1 &  1 &  1   \\
$b$ &1 &1&    0.05 & 0.05         \\
$\beta_1$  &0   &0  & 0 &   0    \\
$\beta_2$  &0   &0  & 0.2 &  0.2 \\
$\beta_3$  &0    &0  & -0.01 &  -0.01\\
$\beta_4^1$ &0  &0 &  3 &  3\\
$\beta_4^2$ &0  &0 &   2 &  2\\
$\beta_4^3$ &0  &0 &   0 & 0\\
$\beta_5$ &0    &0   & -0.5 & -0.5 \\
$\beta_{\eta_0}$ &0 &0 & 0 & 0 \\
$\beta_{\eta_1}$ & 0  & 0 & 0& 0\\
$\tau_0^1$ &0   &0  & 0& 0\\
$\tau_0^2$ &0   &0  & 0& 0\\
$\tau_0^3$ &0   &0  & 0& 0\\
$\gamma_{j_0}$ &0  &0 & 0& 0\\
$\gamma_{j_1}$ &0  &0 &  0& 0\\
$\gamma_1$ &0  &0 &    0& 0\\
$\gamma_2$ &0  &0 &   0& 0\\
$\psi_0$  &0.05  &0.05 & 0.05  & 0.05\\
$\delta$  &0.95   & 0.95 & 0.95 & 0.95  \\
$\sigma$ & 1 & 1 & 1 & 1 \\
Travel time & 1 for all & Actual & Actual & Counterfactual \\
\bottomrule
\end{tabular}
\end{table}

\section{Simulation scenarios}
 \textbf{Scenario 1:} Using these coefficients, I solve the model for CCPs and the equilibrium price vector, where region 1 is numeraire. All regions are completely homogeneous both in terms of wages, commuting costs and amenities. When solving for the equilibrium the initialization of states make a difference. In all the presented scenarios, I let all state variables be equally distributed across its values and let all regions have the same distributions of states. Locations are therefore also homogeneous initially with 1/3 of the individuals residing in each region, 25\% working in region 1, 2, 3 and 4, respectively, 50\% of the agents are highly educated within each region and so on. This scenario is therefore mainly used as a check that the model is coded correctly and as a benchmark for the remaining cases. One should expect to see a flat CCP in terms of both $rh$ and $rw$, i.e. equal probabilities on each if its values. This is indeed the case, cf. \autoref{fig:sce1}. As expected, the relative prices of housing in all regions are all 1, cf. \autoref{tab:eqp}. Again, this is due to the homogeneity of all states and parameter values across regions.

\begin{figure}
\centering
\includegraphics[width=0.7\textwidth, trim=3 3 3 3,clip]{ccp_eq1.png} 
\caption{Model solution of CCPs (scenario 1)}
\label{fig:sce1}
\end{figure}

\begin{table}[]
\centering
\caption{Equilibrium prices}
\label{tab:eqp}
\begin{tabular}{@{}l r r r r@{}}
\toprule
&Case 1 & Case 2 & Case 3      & Case 4        \\ \midrule
$P^2$     & 1      & 1           & 1           & 1 \\
$P^2$     & 1      & 0.981 & 0.925 &  0.882\\
$P^3$     & 1      & 0.700 & 0.390 &  0.663 \\ \bottomrule
\end{tabular}
\end{table}

\textbf{Scenario 2}: Since I want to explore how different commuting time affects housing prices, I now use the travel times from the LTM data as it appears in \autoref{tab:timedata} rather than an equal travel time across and within all regions. I solve for the new equilibrium, keeping the parameters constant. As evident from \autoref{fig:sce2}, the CCPs for work location now change. There is a much higher probability of choosing $rw=4$, i.e. unemployment, since this is associated with a 0 commute and a base income of 1 just like the other work regions. $rw=3$ is particularly unlikely to be chosen since its average commute is very long. From the left top corner we see that nothing has changed for the CCPs for $rh$. One might conjecture that since travel times are now diverse and some work locations are more preferred than others, people would tend to also live in these work regions. The reason this is not the case is due to the equilibrating house prices. These ensure that the excess demand of housing in each region is 0. This requires that the expected share of individuals who move away from a region equals the expected share that moves in. Since there is no outside option in the model so far, e.g. renting instead of owning or a construction sector, one cannot move to a region without buying a house from another person who moves out. This implies that CCPs for home regions are unaffected by changes in parameters and over time. On the other hand, the prices do indeed change, cf. column 2 of \autoref{tab:eqp}. For given prices from scenario 1, there is increased demand for living in region 1 since within and to and from this region there are relatively low travel times. However, this excess supply from region 2 and 3 drives down the prices of these two regions until people living here are fully compensated for the disutility they incur from paying the commuting costs. As region 3 has even higher travel costs on average than region 2, prices in this region decrease the most to 0.70. Whereas all 12 different choice combinations were equally likely in scenario 1, choice 12 corresponding to living in region 3 and being unemployed is now much more likely than the others. On the other hand, working in region 1 and 2 when living in region 3 is not very likely. The reason that this exact combination of choice 12 has a relatively high probability is that when working in region 3 it is associated with a much higher utility if not working compared to working, since any work involves long commute. Also choices 4 and 8 have a relatively high CCP. Choice 4 involves residing in region 1 and again being unemployed and the same for choice 8 except that one lives in region 2. For home regions 1 and 2 there is not as high a disutility of working since the commute is shorter. This explains why the CCPs across work regions are not as distinct for these two residential locations. However, for both regions working in region 3 is unlikely, again due to the long travel time and only people with very good taste shocks for combinations involving work will do so since there is no difference in deterministic income across work choices. 

\renewcommand{\arraystretch}{0.5}
\begin{table}[]
\centering
\caption{Travel times (minutes) to and from regions}
\label{tab:timedata}
\begin{tabular}{@{}l c c c @{}} 
\toprule
                    & Copenhagen city & Copenhagen surroundings & Rest of Zealand \\
\midrule
Copenhagen city         & 6.9        & 14.9               & 43.7     \\
Copenhagen surroundings & 14.9        & 10.8                & 35.6    \\
Rest of Zealand         & 43.7    & 35.6             & 37.9     \\ \bottomrule
\end{tabular}
\end{table}


\begin{figure}
\centering
\begin{minipage}{0.7\textwidth}
\includegraphics[width=\linewidth, trim=3 3 3 3,clip]{ccp_eq2.png} 
{\footnotesize \emph{Note:} (Home,Job) indices are: 1 = (1,1), 2 = (1,2), 3 = (1,3), 4 = (1,4), 5 = (2,1), 6 = (2,2), 7 = (2,3), 8 = (2,4), 9 = (3,1), 10 = (3,2), 11 = (3,3) and 12 = (3,4). \par}
\end{minipage}
\caption{Model solution of CCPs (scenario 2)}
\label{fig:sce2}
\end{figure}

\textbf{Scenario 3:} Continuing to scenario 3, I specify an income equation that differs across regions. Base wages are now higher in region 1 than the rest of Zealand, while region 2 has higher wages than region 3. I let unemployment benefits be very low. On top of the region-specific base level wages, there is also an age profile. Given the second order polynomial and the coefficients, there is a positive, but decreasing return to age until age 10. Hereafter the return becomes negative, but that is beyond the time horizon of the model. Since people seem to sort according to education, cf. \autoref{sec:descriptives}, I let the return to high education be 3, 2 and 0 for each of the work regions, respectively. In region 1 and 2 there are thus very high returns to being highly educated compared to the base wage contrary to region 3 where there is no such gain. Moreover, there is now a penalty from being unemployed in the previous period of -0.5 in current period income. The unconditional CCP of each work region appears from \autoref{fig:sce3}. The unemployment choice is no longer the favorite one. Both because it is now associated with a lower income and because it harms the potential income in the next period. However, working in region 3 has become very implausible. That is due to the fact that, again, the commuting time is long and the wages are not very good in this region relative to both regions 1 and 2 and unemployment. On the other hand, region 1 is now the preferred place to work due to both the higher base wage, but also because of the higher return to education (concerns those highly educated only) while at the same time not having to travel for a long time. The discrepancy between the CCP of working in region 1 and 2 has also risen. Since travel times have been kept constant, this is solely due to the higher base wage in region 1. We also see from the lower left corner that the distribution across choice combinations has been altered. Interestingly, region 2 is now no longer the preferred work region when living in region 2. The higher wage in region 1 compensates for the extra commute one has to do. The same holds for region 3, where both regions 1 and 2 are preferred over region 3 itself. Nevertheless, unemployment still has the highest associated value. Considering the effect on equilibrium prices, column 3 in \autoref{tab:eqp} reveals that demand decreased a lot for region 3 and a little for region 2 given the scenario 2 prices. People dislike residing in region 3 even more now where wages are so low compared to the remaining work alternatives. People therefore must be compensated for having to commute even longer in order to get a fair wage and get that by having lower housing costs. The same mechanism is brought into play for region 2 where people are now inclined to move to region 1 to avoid the longer commute and benefit from the higher wage. This causes excess supply in these regions implying their prices to fall relatively to region 1. 

Looking more into the sorting mechanisms, \autoref{fig:sce3s} conditions on having either a low (s=0) or high education (s=1). By comparing the two figures in the upper part it is clear that highly educated individuals are more likely to work in region 1 compared to those with low education. This is not surprising, since the marginal benefit from doing so is higher for these individuals. The same holds for region 2. Since there is no return to education in region 3, there is only a very small difference in the CCP of choosing that work region. Still, there is a slightly higher probability for low educated individuals to reside in region 3, cf. the lower parts of the figure. The cause of that is that a larger share of this crowd are unemployed. When unemployed it does not matter where you live, why living in region 3 which otherwise is characterized by a long commute, is not as bad as it is for the average highly educated people. It matches empirical facts well that people with low education tend to live outside Copenhagen (except for a large part of Northern Zealand which is also part of region 3), cf. \autoref{sec:descriptives}. Also the fact that these people have a higher (voluntary) unemployment probability is in line with empirics. The current model explains that by imposing that the value of being unemployed is not \textit{much} worse than having a job, since the wage is quite low in any case. This results feeds into the debate about how to set the level of unemployment benefits to ensure that unemployed persons actually have an incentive to take a job. From the lower part of the figure, it is evident that there is a slight discrepancy in CCPs of each home region across education groups besides that observed for region 3. Since region 1 is the preferred work region for highly educated, all else equal, they also tend to live here to benefit from the lower commute to the job locations they actually tend to have. The differences are very small though and much smaller than seen in data. This points to the chosen parameter values not being representative of the real world while it is probably also an artifact of the very stylized framework. What can be concluded from this conditional analysis is that the model predicts sorting patterns that do not conflict with what we see in real data. It also explains why home and job locations are tied together; namely because people dislike commuting. 

\begin{figure}
\centering
\begin{minipage}{0.7\textwidth}
\includegraphics[width=\linewidth, trim=3 3 3 3,clip]{ccp_eq3.png} 
{\footnotesize \emph{Note:} (Home,Job) indices are: 1 = (1,1), 2 = (1,2), 3 = (1,3), 4 = (1,4), 5 = (2,1), 6 = (2,2), 7 = (2,3), 8 = (2,4), 9 = (3,1), 10 = (3,2), 11 = (3,3) and 12 = (3,4). \par}
\end{minipage}
\caption{Model solution of CCPs (scenario 3)}
\label{fig:sce3}
\end{figure}

\begin{figure}
\centering
\includegraphics[width=0.7\textwidth, trim=3 3 3 3,clip]{ccp_eq3_bys.png} 
\caption{Model solution of CCPs (scenario 3) by education groups}
\label{fig:sce3s}
\end{figure}

\textbf{Scenario 4}: Until now the scenarios have been characterized by altering parameters or travel times for several regions at a time. In scenario 4 I impose a counterfactual commute time from region 3 to region 1 which is 60\% lower than the one calculated from data (the remaining travel times are kept constant at their actual levels). So instead of a commuting time of 44 minutes to Copenhagen city, it is now approx 17 minutes, just above the 14 minutes it takes from Copenhagen surroundings to the city. The idea is to gain intuition about how building better and fast public transport from areas outside Copenhagen to Copenhagen city would affect prices and location decisions according to this stylized framework. From \autoref{fig:sce4} it is clear that the reduced travel time does have quite an effect. First of all region 1 is now very likely to be chosen as the work region, close to 50\% of all individuals choose that one. The increase is mainly at the expense of people not choosing regions 2 and 4, while work region 3 still has a the lowest probability, though it does increase slightly. Consequently, the effect is that people living in region 3 now find it worthwhile to travel to work in region 1, where they can earn a much higher base wage and also get a higher return to education. This can also be seen from the lower right panel, where the choice of working in 1 is now by far the most popular choice for those living in region 3. There are also fewer people working in region 2 even though this region's travel time was not affected. The reason is that there are people who now reside in region 3 who previously wanted to commute some distance to region 1 to get the higher salary but did not want to pay the higher housing costs of living in region 1, why they lived in region 2. However, with the faster connection between 1 and 3, some of these people are now pushed over the threshold for when it is optimal to work at the higher pay in region 1, paying lower house prices in region 3 and commuting a bit longer. From the lower left panel we also see that conditional on living in region 2, the CCPs of working in either of the job locations is unaffected compared to scenario 3. The last column in \autoref{tab:eqp} reveals that compared to scenario 3 the housing prices increase in region 3 and decrease in region 2 compared to 1. The explanation for the higher prices in region 3 is that all else equal, as already mentioned, it has become more popular to reside here due to the lower commute. If politicians seek to avoid depopulation and too big differences in housing prices across regions, the simulation suggests that faster commute might be a useful policy to implement, but of course the size of the effect depends on all the parameter values. Since region 2 is no longer much better in terms of commute to region 1 compared to region 3 there is a downwards pressure on the prices in region 2. One thing to note is that in a static model prices of region 2 would have decreased even more. The reason is that in the dynamic model people take into account that by residing in region 2 they can still move to region 3 in the future by selling their relatively expensive house in region 2 and buy a cheaper one in region 3. This tends to keep demand for region 2 at a higher level (the same holds for region 1). Had agents not been dynamic decisions makers they would not care about this potential future capital gain. This mechanism is of course also relevant in the previous scenarios. 

Overall, this section showed that agents behaving according to the model indeed do react to incentives in terms of both wages, return to education and commuting. It also underlined that the very differential prices across regions in Denmark are consistent with there being diverse wage potentials, commuting time and return to education. Under the assumption that the model is a good description of reality, there is thus great scope for politicians to affect the degree of urbanization, depopulation and the very differential trends for house prices in certain areas of Denmark as recently discussed in the media. Nevertheless, the main goal of the section was to demonstrate that the model indeed is able to make valuable predictions.

\begin{figure}
\centering
\begin{minipage}{0.7\textwidth}
\includegraphics[width=\linewidth, trim=3 3 3 3,clip]{ccp_eq4.png} 
{\footnotesize \emph{Note:} (Home,Job) indices are: 1 = (1,1), 2 = (1,2), 3 = (1,3), 4 = (1,4), 5 = (2,1), 6 = (2,2), 7 = (2,3), 8 = (2,4), 9 = (3,1), 10 = (3,2), 11 = (3,3) and 12 = (3,4). \par}
\end{minipage}
\caption{Model solution of CCPs (scenario 4)}
\label{fig:sce4}
\end{figure}